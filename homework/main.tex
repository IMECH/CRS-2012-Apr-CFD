\documentclass[12pt,a4paper,boxed]{caspset}

% set 1-inch margins in the document
%\usepackage[left=1in,right=1in,top=1.2in,bottom=1in]{geometry}
\usepackage[left=1in,right=1in,top=1.2in,bottom=1in]{geometry}
\usepackage{lastpage}
\usepackage{pdflscape}
% include this if you want to import graphics files with /includegraphics
\usepackage{multicol} 
\usepackage{capt-of}%%To get the caption
\usepackage{graphicx}
\usepackage{lscape}
\usepackage{amsmath,amsfonts,amsthm,amssymb}
\usepackage{mathtools}
\usepackage{hyperref}
\usepackage{setspace}
\usepackage{fancyhdr}
\usepackage{lastpage}
\usepackage{extramarks}
\usepackage{chngpage}
\usepackage{soul}
\usepackage[usenames,dvipsnames]{color}
\usepackage{graphicx,float,wrapfig}
\usepackage{ifthen}
\usepackage{listings}
\usepackage{courier}
\usepackage{multimedia}
\usepackage[toc,page,title,titletoc,header]{appendix}
\usepackage{color, soul}
\usepackage{tikz}
\usepackage{array}
\usepackage{multirow}
\usepackage{todonotes}
\usepackage{pdfpages}
\usepackage{pgfplots}
\usetikzlibrary{%
    decorations.pathreplacing,%
    decorations.pathmorphing,arrows
}
\usetikzlibrary{calc}
\usepgfplotslibrary{polar}
%%%%%%%%%%%%%%%%%%%%%%%%%%%%%%%%%%%%%%%%%%%%%%%%%%%%%%
\usepackage{xeCJK}
%\usepackage{fontspec}
\setCJKmainfont[BoldFont=simhei.ttf]{simsun.ttf}
%\setCJKsansfont{simhei.ttf}
%\setCJKmonofont{simfang.ttf}

%\setCJKmainfont{Adobe Song Std}
%\setCJKmainfont[BoldFont=Adobe Heiti Std]{Adobe Song Std}
%%%%%%%%%%%%%%%%%%%%%%%%%%%%%%%%%%%%%%%%%%%%%%%%%%%%%%

\graphicspath{{figures/}}

\setulcolor{red}

\setlength{\marginparwidth}{1in}

\newcommand{\hmwkTitle}{高等计算流体力学}
\newcommand{\hmwkSubTitle}{高等计算流体力学课程家庭作业题(A2)}
\newcommand{\hmwkDueDate}{\today}
\newcommand{\hmwkClass}{物理学院}
\newcommand{\hmwkClassTime}{Tue/Thu.{~}09:50}
\newcommand{\hmwkClassInstructor}{张德良}
\newcommand{\hmwkAuthorName}{周吕文}


\hypersetup{pdfauthor={\hmwkAuthorName}, 
            pdftitle={高等计算流体力学课程家庭作业题(A2)}, 
            pdfsubject={\hmwkTitle, \hmwkClassInstructor},
            pdfkeywords={高等计算流体力学},
            pdfproducer={XeLateX with hyperref},
            pdfcreator={Xelatex}}

%% Setup the header and footer
\pagestyle{fancy}                                                       %
\lhead{\hmwkAuthorName}                                                 %
\chead{\hmwkClass\ (\hmwkClassInstructor): \hmwkTitle}  %
\rhead{第\ \thepage\ 页,{~} 共\ \protect\pageref{LastPage} 页}          %                                %
\definecolor{DarkGreen}{rgb}{0.0,0.45,0.0}

%%%%%%%%%%%%%%%%%%%%%%%%%%%%%%%%%%%%%%%%%%%%%%%%%%%%%%%%%%%%%



\makeatletter
\newcommand{\rmnum}[1]{\romannumeral #1}
\newcommand{\Rmnum}[1]{\expandafter\@slowromancap\romannumeral #1@}
\makeatother

\renewcommand\refname{\bf\large 参考文献}
\renewcommand\contentsname{\bf 目 \ \ \ 录}
\renewcommand\figurename{\bf 图}
\renewcommand\tablename{\bf 表}
\renewcommand{\appendixtocname}{附录}
\renewcommand{\appendixpagename}{附录}
\renewcommand\listfigurename{图目录}

\name{周吕文{~}201128000718065}
\class{物理学院{~}20110308班}
\assignment{作业 \# A2}
\duedate{4/15/2012}

\newcommand\invisiblesection[1]{%
  \refstepcounter{section}%
  \addcontentsline{toc}{section}{\protect\numberline{\thesection}#1}%
  \sectionmark{#1}}

\newcommand\invisiblesubsection[1]{%
  \refstepcounter{subsection}%
  \addcontentsline{toc}{subsection}{\protect\numberline{\thesubsection}#1}%
  \subsectionmark{#1}}


\newcommand{\matlabscript}[2]
{\definecolor{MyDarkGreen}{rgb}{0.0,0.3,0.0}
\definecolor{hellgelb}{rgb}{0.96,0.96,0.96}
\definecolor{DarkPurple}{rgb}{0.6,0,0.4}
    \lstset{%
       language=Matlab,                        % Use MATLAB
        frame=single,                           % Single frame around code
        basicstyle=\footnotesize\ttfamily,    
        keywordstyle=[1]\color{blue}, 
        keywordstyle=[2]\color{DarkPurple}, 
        keywordstyle=[3]\color{blue}\underbar, 
        identifierstyle=,  
        commentstyle=\color{MyDarkGreen}\footnotesize,
        stringstyle=\color{DarkPurple}, 
        showstringspaces=false,            
        tabsize=5,     
        morekeywords={xlim,ylim,var,alpha,factorial,poissrnd,normpdf,normcdf},
        morekeywords=[2]{on, off, interp},
        morecomment=[l][\color{blue}]{...},    
     columns=fixed,
     tabsize=4,%
     frame=single,%
     framerule=1pt,
     extendedchars=true,%
     showspaces=false,%
     showstringspaces=false,%
     numbers=left,%
     numberstyle=\tiny\ttfamily,%
     numbersep=1em,%
     breaklines=true,%
     breakindent=10pt,%
     backgroundcolor=\color{hellgelb},%
     breakautoindent=true,%
     captionpos=t,%
     xleftmargin=1em,%
     xrightmargin=\fboxsep%
    }
\lstinputlisting[caption=#2,label=#1]{#1.m}}


\begin{document}
%\title{摄动理论\&渐进方法(2012春)作业解答\\复习材料\\ \vspace{-20pt} }

%\author{{\Large 授课老师:周显初, 李家春}\\ \vspace{50pt}\\ 周吕文\\ \href{mailto:zhou.lv.wen@gmail.com}{zhou.lv.wen@gmail.com}\\ \vspace{50pt}}
%\date{中国科学院力学研究所\\2012年06月05日}
%\maketitle 
%\enlargethispage{1.5\baselineskip}
%\noindent{\small\textbf{说明}: 本文档是由本人的摄动理论\&渐进方法作业整理而成. 目录中标题为\textcolor{blue}{\bf 蓝色}的个人以为比较重要. 时间有限, 难免有误, 发现问题, 请电邮我. 祝各位考试顺利.}
%\vspace{-1em}
%\tableofcontents
%\setcounter{page}{0}
%\newpage

%\includepdf[pages=1-]{cover}

%\invisiblesection{第一章{~}渐进级数}
%\problemlist{\bf 第一章{~}渐进级数}
%\invisiblesubsection{\textcolor{blue}{习题1.1}}
\problemlist{高等计算流体力学课程家庭作业题(A2)}
\begin{problem}[问题1]
把有量纲二维定常抛物化N-S方程组:
\begin{equation}\label{eq:NS01}
u\frac{\partial\rho}{\partial x}
+ v\frac{\partial\rho}{\partial y}
+ \rho\Big(\frac{\partial u}{\partial x} + \frac{\partial v}{\partial y}\Big)
=0
\end{equation}
\begin{equation}\label{eq:NS02}
u\frac{\partial u}{\partial x} + v\frac{\partial u}{\partial y}
+ \frac{1}{\rho}\frac{\partial p}{\partial x} = \frac{\mu}{\rho}\frac{\partial^2 u}{\partial y^2}
\end{equation}
\begin{equation}\label{eq:NS03}
u\frac{\partial v}{\partial x} + v\frac{\partial v}{\partial y}
+\frac{1}{\rho}\frac{\partial p}{\partial y}
=\frac{4}{3}\frac{\mu}{\rho}\frac{\partial^2v}{\partial y^2}
\end{equation}
\begin{equation}\label{eq:NS04}
u\frac{\partial T}{\partial x} + v\frac{\partial T}{\partial y}
+(\gamma-1)T\Big(\frac{\partial u}{\partial x} + \frac{\partial v}{\partial y }\Big)
=\frac{\gamma - 1}{\rho R}N + \frac{\gamma-1}{\rho R}\frac{\partial}{\partial y}\Big(k\frac{\partial T}{\partial y}\Big)
\end{equation}
转化成无量纲形式. 其中$N=\mu [ (\frac{\partial u}{\partial y})^2 + \frac{4}{3}(\frac{\partial v}{\partial y})^2]$.
\end{problem}

\begin{solution}
引入参考量: 流场中物体特征长度$L_{*}$, 自由来流在$x$方向的速度$u_{*}$, 自由来流的密度$\rho_{*}$和温度$T_{*}$,
自由来流等熵声速$a_{*}$, 标准状态下重力加速度$g_{*}$. 由此可得定义如下各量纲为一的流动量:
\[
\overline{x}=\frac{x}{L_{*}},~~
\overline{y}=\frac{y}{L_{*}},~~
\overline{u}=\frac{u}{u_{*}},~~
\overline{v}=\frac{v}{u_{*}},~~
\overline{\rho}=\frac{\rho}{\rho_{*}},~~
\overline{p}=\frac{p}{\rho_{*}u_{*}^{2}},~~
\overline{\mu}=\frac{\mu}{\mu_{*}},~~
\overline{T}=\frac{T}{T_{*}}
\]
因此有
\begin{equation}\label{eq:NS05}
x = \overline{x}L_{*},~
y = \overline{y}L_{*},~
u = \overline{u}u_{*},~
v = \overline{v}u_{*},~
\rho =\overline{\rho}\rho_{*},~
p = \overline{p}\rho_{*}u_{*}^{2},~
\mu = \overline{\mu}\mu_{*},~
T=\overline{T}T_{*}
\end{equation}
下面分别对题中四式无量纲化:
\begin{itemize}
\item 将式(\ref{eq:NS05})中的各式代入(\ref{eq:NS01})得
\[
\overline{u}u_{*}\frac{\partial\overline{\rho}\rho_{*}}{\partial\overline{x}L_{*}}+\overline{v}u_{*}\frac{\partial\overline{\rho}\rho_{*}}{\partial\overline{y}L_{*}}+\overline{\rho}\rho_{*}\Big(\frac{\partial\overline{u}u_{*}}{\partial\overline{x}L_{*}}+\frac{\partial\overline{v}u_{*}}{\partial\overline{y}L_{*}}\Big) = 0
\]
整理得
\[
\frac{u_{*}\rho_{*}}{L_{*}}\Big[\overline{u}\frac{\partial\overline{\rho}}{\partial\overline{x}}+\overline{v}\frac{\partial\overline{\rho}}{\partial\overline{y}}+\overline{\rho}\Big(\frac{\partial\overline{u}}{\partial\overline{x}}+\frac{\partial\overline{v}}{\partial\overline{y}}\Big)\Big]=0
\]
由于$u_*\rho_*/L_*\neq 0$, 因此可得到(1)式的无量纲形式
\[
\overline{u}\frac{\partial\overline{\rho}}{\partial\overline{x}}+\overline{v}\frac{\partial\overline{\rho}}{\partial\overline{y}}+\overline{\rho}\Big(\frac{\partial\overline{u}}{\partial\overline{x}}+\frac{\partial\overline{v}}{\partial\overline{y}}\Big)=0
\]

\item 将式(\ref{eq:NS05})中的各式代入(\ref{eq:NS02})得
\[
\overline{u}u_{*}\frac{\partial\overline{u}u_{*}}{\partial\overline{x}L_{*}}+\overline{v}u_{*}\frac{\partial\overline{v}u_{*}}{\partial\overline{y}L_{*}}+\frac{1}{\overline{\rho}\rho_{*}}\frac{\partial\overline{p}\rho_{*}u_{*}^{2}}{\partial\overline{x}L_{*}}
=\frac{\overline{\mu}\mu_{*}}{\overline{\rho}\rho_{*}}\frac{\partial^{2}\overline{u}u_{*}}{\partial(\overline{y}L_{*})^{2}}
\]
整理得
\[
\frac{u_{*}^{2}}{L_{*}}\Big[\overline{u}\frac{\partial\overline{u}}{\partial\overline{x}}+\overline{v}\frac{\partial\overline{v}}{\partial\overline{y}}+\frac{1}{\overline{\rho}}\frac{\partial\overline{p}}{\partial\overline{x}}\Big]
=\frac{\overline{\mu}\mu_{*}u_{*}}{\overline{\rho}\rho_{*}L_{*}^{2}}\frac{\partial^{2}\overline{u}}{\partial\overline{y}^2}
\Rightarrow
\overline{u}\frac{\partial\overline{u}}{\partial\overline{x}}+\overline{v}\frac{\partial\overline{v}}{\partial\overline{y}}+\frac{1}{\overline{\rho}}\frac{\partial\overline{p}}{\partial\overline{x}}
=\frac{\overline{\mu}\mu_{*}}{\overline{\rho}\rho_{*}u_{*}L_{*}}\frac{\partial^{2}\overline{u}}{\partial\overline{y}^2}
\]
其中$\frac{\mu_{*}}{\rho_{*}u_{*}L_{*}}=\frac{1}{Re}$, Re为雷诺数.
代入上式便可得到(\ref{eq:NS02})式的无量纲形式
\[
\overline{u}\frac{\partial\overline{u}}{\partial\overline{x}}+\overline{v}\frac{\partial\overline{v}}{\partial\overline{y}}+\frac{1}{\overline{\rho}}\frac{\partial\overline{p}}{\partial\overline{x}}
=\frac{\overline{\mu}}{\overline{\rho}\textrm{Re}}\frac{\partial^{2}\overline{u}}{\partial\overline{y}^2}
\]
\item 与(\ref{eq:NS02})式的无量纲过程类似, 可得到(\ref{eq:NS03})式的无量纲形式
\[
\overline{u}\frac{\partial\overline{v}}{\partial\overline{x}}
+\overline{v}\frac{\partial\overline{v}}{\partial\overline{y}}
+\frac{1}{\overline{\rho}}\frac{\partial\overline{p}}{\partial\overline{y}}
=\frac{4\overline{\mu}}{\overline{3\rho}\textrm{Re}}\frac{\partial^{2}\overline{v}}{\partial\overline{y}^2}
\]
\item 将式(\ref{eq:NS05})中的各式代入(\ref{eq:NS04})中并整理得
\[
\frac{u_{*}T_{*}}{L_{*}}\Big[\overline{u}\frac{\partial\overline{T}}{\partial\overline{x}}
+\overline{v}\frac{\partial\overline{T}}{\partial\overline{y}}
+(\gamma-1)\overline{T}\Big(\frac{\partial\overline{u}}{\partial\overline{x}}
+\frac{\partial\overline{v}}{\partial\overline{y}}\Big)\Big]
=
\frac{\mu_{*}u_{*}^2}{\rho_{*}L_{*}^{2}}
\frac{\gamma-1}{\overline{\rho}R}
\overline{N}
+\frac{k_*T_{*}}{\rho_{*}L_{*}^{2}}
\frac{\gamma-1}{\overline{\rho}R}
\frac{\partial}{\partial\overline{y}}
\Big(\overline{k}\frac{\partial\overline{T}}{\partial\overline{y}}\Big)
\]
其中$\overline{N}=\overline{\mu}
[(\frac{\partial\overline{u}}{\partial\overline{x}})^{2}
+\frac{4}{3}(\frac{\partial\overline{v}}{\partial\overline{y}})^{2}]$. 化简上式得
\[
\overline{u}\frac{\partial\overline{T}}{\partial\overline{x}}
+\overline{v}\frac{\partial\overline{T}}{\partial\overline{y}}
+(\gamma-1)\overline{T}\Big(\frac{\partial\overline{u}}{\partial\overline{x}}
+\frac{\partial\overline{v}}{\partial\overline{y}}\Big)
=
\underbrace{\textcolor{red}{\frac{\mu_{*}u_{*}}{\rho_{*}L_{*}T_{*}}
\frac{\gamma-1}{\overline{\rho}R}
}}_{(a)}\overline{N}
+\underbrace{\textcolor{blue}{\frac{k_*}{\rho_{*}L_{*}u_{*}}
\frac{\gamma-1}{\overline{\rho}R}}}_{(b)}
\frac{\partial}{\partial\overline{y}}
\Big(\overline{k}\frac{\partial\overline{T}}{\partial\overline{y}}\Big)
\]
对于(a), 注意到其中$R=\frac{p}{\rho T}=\frac{\overline{p}}{\overline{\rho}\overline{T}}\frac{u_*^2}{T_*}$, 因此可令$R=\overline{R}u_*^2/T_*$. 上式中的(a)项可化为
\[
\textcolor{red}{\frac{\mu_{*}u_{*}}{\rho_{*}L_{*}T_{*}}
\frac{\gamma-1}{\overline{\rho}R}}
=
\frac{\mu_{*}u_{*}T_{*}}{\rho_{*}L_{*}T_{*}}
\frac{\gamma-1}{\overline{\rho}\overline{R}u_{*}^2}
=
\frac{\mu_{*}}{\rho_{*}L_{*}u_{*}}
\frac{\gamma-1}{\overline{\rho}\overline{R}}
=
\frac{\gamma-1}{\textrm{Re} \overline{\rho}\overline{R}}
\]
对于(b)项, 注意到$R=C_p-C_v$, $\gamma=C_p/C_v$, $\gamma - 1 = (C_p -C_v)/C_v$. 因此(b)项可化为
{\setlength\arraycolsep{2pt}
\begin{eqnarray}
\textcolor{blue}{\frac{k_*}{\rho_{*}L_{*}u_{*}}
\frac{\gamma-1}{\overline{\rho}R}}
&=&
\frac{k_*}{\rho_{*}L_{*}u_{*}}
\frac{(C_p -C_v)/C_v}{\overline{\rho}(C_p-C_v)}
=
\frac{k_*}{\rho_{*}L_{*}u_{*}}
\frac{1}{C_v\overline{\rho}}
\nonumber\\
&=&
\frac{k_*}{\rho_{*}L_{*}u_{*}}
\frac{C_p}{C_p C_v\overline{\rho}}
=
\frac{\mu_*}{\rho_{*}u_{*}L_{*}}
\frac{k_*}{\mu_*C_p}
\frac{\gamma}{\overline{\rho}}
\nonumber\\
&=&
\frac{1}{\textrm{Re}}
\frac{1}{\textrm{Pr}}
\frac{\gamma}{\overline{\rho}}\nonumber
\end{eqnarray}}
因此, 最终可得(\ref{eq:NS04})式的无量纲形式
\[
\overline{u}\frac{\partial\overline{T}}{\partial\overline{x}}
+\overline{v}\frac{\partial\overline{T}}{\partial\overline{y}}
+(\gamma-1)\overline{T}\Big(\frac{\partial\overline{u}}{\partial\overline{x}}
+\frac{\partial\overline{v}}{\partial\overline{y}}\Big)
=
\frac{\gamma-1}{\overline{\rho}~\textrm{Re}~\overline{R}}
\overline{N}+
\frac{\gamma}{\overline{\rho}~\textrm{Pr}~\textrm{Re}}
\frac{\partial}{\partial\overline{y}}
\Big(\overline{k}\frac{\partial\overline{T}}{\partial\overline{y}}\Big)
\]
其中$\overline{N}=\overline{\mu}
[(\frac{\partial\overline{u}}{\partial\overline{x}})^{2}
+\frac{4}{3}(\frac{\partial\overline{v}}{\partial\overline{y}})^{2}]$.
\end{itemize}
为了方便, 常常把量纲为一的量上的上画线符号略去, 最终得到二维定常抛物化N-S方程组的无量纲形式:
\[
{u}\frac{\partial{\rho}}{\partial{x}}+{v}\frac{\partial{\rho}}{\partial{y}}+{\rho}\Big(\frac{\partial{u}}{\partial{x}}+\frac{\partial{v}}{\partial{y}}\Big)=0
\]
\[
{u}\frac{\partial{u}}{\partial{x}}+{v}\frac{\partial{v}}{\partial{y}}+\frac{1}{{\rho}}\frac{\partial{p}}{\partial{x}}
=\frac{{\mu}}{{\rho}\textrm{Re}}\frac{\partial^{2}{u}}{\partial{y}^2}
\]
\[
{u}\frac{\partial{v}}{\partial{x}}
+{v}\frac{\partial{v}}{\partial{y}}
+\frac{1}{{\rho}}\frac{\partial{p}}{\partial{y}}
=\frac{4{\mu}}{{3\rho}\textrm{Re}}\frac{\partial^{2}{v}}{\partial{y}^2}
\]
\[
{u}\frac{\partial{T}}{\partial{x}}
+{v}\frac{\partial{T}}{\partial{y}}
+(\gamma-1){T}\Big(\frac{\partial{u}}{\partial{x}}
+\frac{\partial{v}}{\partial{y}}\Big)
=
\frac{\gamma-1}{{\rho}~\textrm{Re}~{R}}
{N}+
\frac{\gamma}{{\rho}~\textrm{Pr}~\textrm{Re}}
\frac{\partial}{\partial{y}}
\Big({k}\frac{\partial{T}}{\partial{y}}\Big)
\]
其中${N}={\mu}
[(\frac{\partial{u}}{\partial{x}})^{2}
+\frac{4}{3}(\frac{\partial{v}}{\partial{y}})^{2}]$.
\end{solution} 

\begin{problem}[问题2]
一维非定常粘性流动N-S方程组
\[
\mathbf{u_t} + \mathbf{f_x} = \mathbf{s_x}
\]
其中
\[
\mathbf{u} =
\left[
  \begin{array}{c}
    \rho \\
    \rho u\\
    E \\
  \end{array}
\right], ~~
\mathbf{f} =
\left[
  \begin{array}{c}
    \rho u \\
    \rho u^2 + p \\
    u(E+p) \\
  \end{array}
\right],~~
\mathbf{s} =
\left[
  \begin{array}{c}
    0 \\
    \frac{4}{3}\frac{\mu}{\textrm{Re}}\frac{\partial u}{\partial x} \\
    \frac{4}{3}\frac{\mu u}{\textrm{Re}}\frac{\partial u}{\partial x} + \frac{c_pK}{\textrm{Pr}~\textrm{Re}}\frac{\partial T}{\partial x} \\
  \end{array}
\right]
\]
这里的状态方程为:
\[
p = (\gamma -1)\rho e = (\gamma - 1)\Big(E-\frac{1}{2}\rho u^2\Big)
\]
求出该方程组的特征根, 并分析它的数学性质和类型.
\end{problem}

\begin{solution}
由状态方程可得
\[
E=c_{v}T+\frac{1}{2}\rho u^{2},\: u(E+p)=u\Big(c_{p} T+\frac{1}{2}\rho u^{2}\Big)
\]
因此上述守恒型方程组可改写成非守恒型方程组, 并消去能量和压强消项:
\[
\textcolor{red}{\frac{\partial\rho}{\partial t}}
+\textcolor{blue}{u\frac{\partial\rho}{\partial x}}
+\textcolor{DarkGreen}{\rho\frac{\partial u}{\partial x}} = 0
\]
\[
\textcolor{red}{\frac{\partial u}{\partial t}}
+\textcolor{DarkGreen}{u\frac{\partial u}{\partial x}}
+\textcolor{blue}{\frac{(c_p-c_v)}{\rho}\frac{\partial T}{\partial x}}
=\textcolor{blue}{\frac{4}{3}\frac{\mu}{\rho\textrm{Re}}\frac{\partial}{\partial x}\Big(\frac{\partial u}{\partial x}\Big)}
\]
\begin{multline}
\textcolor{red}{c_{v}\frac{\partial T}{\partial t}}
+\textcolor{DarkGreen}{\rho u\frac{\partial u}{\partial t}}
+\textcolor{DarkGreen}{\frac{1}{2}u^{2}\frac{\partial\rho}{\partial t}}
+\textcolor{DarkGreen}{c_{p} u\frac{\partial T}{\partial x}}
+\textcolor{DarkGreen}{c_{p} T\frac{\partial u}{\partial x}}
+\textcolor{DarkGreen}{\frac{1}{2}u^{3}\frac{\partial\rho}{\partial x}}
+\textcolor{DarkGreen}{\frac{3}{2}\rho u^{2}\frac{\partial u}{\partial x}}\\
\left. =
\textcolor{DarkGreen}{\frac{4}{3}\frac{\mu}{\textrm{Re}}\Big(\frac{\partial u}{\partial x}\Big)^{2}}
+\textcolor{DarkGreen}{\frac{4}{3}\frac{\mu}{\textrm{Re}}u\frac{\partial}{\partial x}\Big(\frac{\partial u}{\partial x}\Big)}+
\textcolor{blue}{\frac{c_{p}}{\textrm{Pr}}\frac{K}{\textrm{Re}}\frac{\partial}{\partial x}\Big(\frac{\partial T}{\partial x}\Big)}\right.\nonumber
\end{multline}
对上述非守恒型方程组进行适当变换, 令
\[
\frac{\partial u}{\partial x} = u_x, ~~ \frac{\partial T}{\partial x} = T_x
\]
可简化为下式
\[
\mathbf{A}\frac{\partial \mathbf{Z}}{\partial t} + \mathbf{B}\frac{\partial \mathbf{z}}{\partial x} = \mathbf{H}
\]
其中
\[
\mathbf{\mathbf{Z}}=[\rho\: u\: T\: u_{x}\: T_{x}]^{\textrm{T}}
\]
\[
A=\left[
\begin{array}{ccccc}
\textcolor{red}{1} & 0 & 0 & 0 & 0\\
0 & \textcolor{red}{1} & 0 & 0 & 0\\
0 & 0 & \textcolor{red}{c_{v}} & 0 & 0\\
0 & 0 & 0 & 0 & 0\\
0 & 0 & 0 & 0 & 0
\end{array}
\right]
\:\:\:\:\:\:\:\:\:\:\:\:\:\:\:\:\:\: \:\:\:\:\:\:\:\:\:\:\:\:\:\:\:\:\:\:\:\:\:\:\:
B=\left[
\begin{array}{ccccc}
\textcolor{blue}{u} & 0 & 0 & 0 & 0\\
0 & 0 & \scriptstyle \textcolor{blue}{\frac{(c_{p}-c_{v})}{\rho}} & \scriptstyle \textcolor{blue}{\frac{-4\mu}{3\rho\textrm{Re}}} & 0\\
0 & 0 & 0 & 0 & \scriptstyle\textcolor{blue}{\frac{-c_{p}K}{\textrm{Pr\textrm{Re}}}}\\
0 & 1 & 0 & 0 & 0\\
0 & 0 & 1 & 0 & 0
\end{array}
\right]
\]
\[
H=\left[\textcolor{DarkGreen}{
\begin{array}{c}
-\rho u_{x}\\
-uu_{x}\\
\frac{4}{3}\frac{\mu}{\textrm{Re}}(u_{x}^{2}+u_{xx})-\frac{3}{2}\rho u^{2}u_{x}-\frac{1}{2}u^{3}\rho_{x}-c_{p}(Tu_{x}+uT_{x})-\rho uu_{t}-\frac{1}{2}u^{2}\rho_{t}\\
u_{x}\\
T_{x}
\end{array}}
\right]
\]
系数矩阵的特征方程为
\[
\textrm{det}(\sigma_1 \mathbf{A} + \sigma_2 \mathbf{B}) = 0
\Rightarrow
(u\sigma_2+\sigma_1)\sigma_2^4=0
\]
它的特征根为:
\[
\lambda_1=u, ~~\lambda_{2,3,4,5} = 0
\]
除实根$\lambda_1$外, 其它均是重根. 由此可知, 该方程组是抛物型的, 但也具有双曲型性质. 从总体上说, 它是抛物型的.
\end{solution} 

\begin{problem}[问题3]
对流方程$\frac{\partial u}{\partial t} + a\frac{\partial u}{\partial x} = 0 ~(a>0)$ 的前半隐差分格式($r=\frac{a\Delta t}{\Delta x}$)为:
\[
u_j^{n+1} = u_j^n -\frac{1}{2}r
\Big[
\big(u_j^n-u_{j-1}^n\big) + \big(u_{j+1}^{n+1}-u_j^{n+1}\big)
\Big]
\]
分析它的精度和稳定性.
\end{problem}

\begin{solution}
\textbf{精度分析:} 将$u_{j}^{n+1}$, $u_{j-1}^{n}$, $u_{j+1}^{n+1}$分别在$u=u_{j}^{n}$展开成泰勒级数
{\setlength\arraycolsep{2pt}
\begin{eqnarray*}
u_{j}^{n+1} & = & u_{j}^{n}+\frac{\partial u}{\partial t}\Delta t\:+\frac{1}{2}\frac{\partial^{2}u}{\partial t^{2}}\Delta t^{2}\:+\frac{1}{6}\frac{\partial^{3}u}{\partial t^{3}}\Delta t^{3}\:+\cdots\\
u_{j-1}^{n} & = & u_{j}^{n}-\frac{\partial u}{\partial x}\Delta x+\frac{1}{2}\frac{\partial^{2}u}{\partial x^{2}}\Delta x^{2}-\frac{1}{6}\frac{\partial^{3}u}{\partial x^{3}}\Delta x^{3}+\cdots\\
%u_{j+1}^{n} & = & u_{j}^{n}+\frac{\partial u}{\partial t}\Delta x+\frac{1}{2}\frac{\partial^{2}u}{\partial t^{2}}\Delta x^{2}+\frac{1}{6}\frac{\partial^{3}u}{\partial t^{3}}\Delta x^{3}+\cdots\\
u_{j+1}^{n+1} & = & u_{j}^{n}+\Big(\Delta x\frac{\partial}{\partial x}+\Delta t\frac{\partial}{\partial t}\Big)u+\frac{1}{2}\Big(\Delta x\frac{\partial}{\partial x}+\Delta t\frac{\partial}{\partial t}\Big)^{2}u+\frac{1}{6}\Big(\Delta x\frac{\partial}{\partial x}+\Delta t\frac{\partial}{\partial t}\Big)^{3}u+\cdots
\end{eqnarray*}}\\
\noindent 因此有
\[
\frac{u_{j+1}^{n+1}-u_{j}^{n+1}}{\Delta x}=\frac{\partial u}{\partial x}+\frac{1}{2}\frac{\partial^{2}u}{\partial x^{2}}\Delta x+\frac{\partial^{2}u}{\partial x\partial t}\Delta t+\frac{1}{6}\frac{\partial^{3}u}{\partial x^{3}}\Delta x^{2}+\frac{1}{2}\frac{\partial^{3}u}{\partial x\partial t^{2}}\Delta t^{2}+\textcolor{magenta}{\frac{1}{2}\frac{\partial^{3}u}{\partial x^{2}\partial t}\Delta x\Delta t}+\cdots
\]
将以上四式代入到差分方程得
{\setlength\arraycolsep{2pt}
\begin{eqnarray*}
 &  & \frac{u_{j}^{n+1}-u_{j}^{n}}{\Delta t}+\frac{a}{2}\Big[\frac{u_{j}^{n}-u_{j-1}^{n}}{\Delta x}+\frac{u_{j+1}^{n+1}-u_{j}^{n+1}}{\Delta x}\Big]\\
 & = & \frac{\partial u}{\partial t}+\textcolor{red}{\frac{1}{2}\frac{\partial^{2}u}{\partial t^{2}}\Delta t}+\textcolor{blue}{\frac{1}{6}\frac{\partial^{3}u}{\partial t^{3}}\Delta t^{2}}+O(\Delta t^{3})+
 \frac{a}{2}\Big(\textcolor[rgb]{0.6,0.3,0.1}{\frac{\partial u}{\partial x}}\textcolor[rgb]{0.33,0.33,0.33}{-\frac{1}{2}\frac{\partial^{2}u}{\partial x^{2}}\Delta x}+\textcolor{DarkGreen}{\frac{1}{6}\frac{\partial^{3}u}{\partial x^{3}}\Delta x^{2}}+O(\Delta x^{3})\Big)\\
 &  & \qquad\qquad+\frac{a}{2}\Big(\textcolor[rgb]{0.6,0.3,0.1}{\frac{\partial u}{\partial x}}+\textcolor[rgb]{0.33,0.33,0.33}{\frac{1}{2}\frac{\partial^{2}u}{\partial x^{2}}\Delta x}+\textcolor{red}{\frac{\partial^{2}u}{\partial x\partial t}\Delta t}+\textcolor{DarkGreen}{\frac{1}{6}\frac{\partial^{3}u}{\partial x^{3}}\Delta x^{2}}+\textcolor{blue}{\frac{1}{2}\frac{\partial^{3}u}{\partial x\partial t^{2}}\Delta t^{2}}+\textcolor{magenta}{O(\Delta x\Delta t)}\Big)\\
 & = & \frac{\partial u}{\partial t}+\textcolor{red}{\frac{1}{2}\Big(\frac{\partial^{2}u}{\partial t^{2}}+a\frac{\partial^{2}u}{\partial x\partial t}\Big)\Delta t}+\textcolor[rgb]{0.6,0.3,0.1}{a\frac{\partial u}{\partial x}}+O(\textcolor{DarkGreen}{\Delta x^{2}}, \textcolor{blue}{\Delta t^2}, \textcolor{magenta}{\Delta x\Delta t})\\
 & = & \frac{\partial u}{\partial t}+a\frac{\partial u}{\partial x}+O(\Delta x^{2},\Delta t^2, \Delta t\Delta x)
\end{eqnarray*}}
因此该差分格式在时间和空间上都是二阶精度.

\vspace{1em}
\noindent\textbf{稳定性分析:} 对差分方程中的各项作Fourier展开$
u_{j}^{n}=\sum_{k=1}^{\infty}\hat{u}_{j}^{n}\mathrm{e}^{\mathrm{i}jk\Delta x}%,\: u_{j}^{n+1}=\sum_{k=1}^{\infty}\hat{u}_{j}^{n+1}\mathrm{e}^{\mathrm{i}jk\Delta x},\: u_{j\pm1}^{n+1}=\sum_{k=1}^{\infty}\hat{u}_{j\pm1}^{n+1}\mathrm{e}^{\mathrm{i}(j\pm1)k\Delta x}
$
并代入原差分方程得到它的每个分量的误差方程:
\[
\hat{u}_{j}^{n+1}\mathrm{e}^{\mathrm{i}jk}=\hat{u}_{j}^{n}\mathrm{e}^{\mathrm{i}jk\Delta x}-\frac{1}{2}r\Big[\Big(\hat{u}_{j}^{n}\mathrm{e}^{\mathrm{i}jk\Delta x}-\hat{u}_{j-1}^{n}\mathrm{e}^{\mathrm{i}(j-1)k\Delta x}\Big)+\Big(\hat{u}_{j+1}^{n+1}\mathrm{e}^{\mathrm{i}(j+1)k\Delta x}-\hat{u}_{j}^{n+1}\mathrm{e}^{\mathrm{i}k\Delta x}\Big)\Big]
\]
%\[
%\frac{\hat{u}_{j}^{n+1}\mathrm{e}^{\mathrm{i}k\Delta x}}{\hat{u}_{j}^{n}\mathrm{e}^{\mathrm{i}k\Delta x}}=1-\frac{1}{2}r\bigg[\Big(1-\frac{\hat{u}_{j-1}^{n}\mathrm{e}^{\mathrm{i}(j-1)k\Delta x}}{\hat{u}_{j}^{n}\mathrm{e}^{\mathrm{i}jk\Delta x}}\Big)+\Big(\frac{\hat{u}_{j+1}^{n+1}\mathrm{e}^{\mathrm{i}(j+1)k\Delta x}}{\hat{u}_{j}^{n}\mathrm{e}^{\mathrm{i}jk\Delta x}}-\frac{\hat{u}_{j}^{n+1}\mathrm{e}^{\mathrm{i}jk\Delta x}}{\hat{u}_{j}^{n}\mathrm{e}^{\mathrm{i}jk\Delta x}}\Big)\bigg]
%\]
放大因子为$G=\hat{u}_{j}^{n+1}/\hat{u}_{j}^{n}$, 则上式可化为
\[
G=1-\frac{1}{2}r\Big[\big(1-\mathrm{e}^{-\mathrm{i}k\Delta x}\big)+G\big(\mathrm{e}^{\mathrm{i}k\Delta x}-1\big)\Big]\,\Longrightarrow\,|G|=\bigg|\frac{2-r+r\mathrm{e}^{-\mathrm{i}k\Delta x}}{2-r+r\mathrm{e}^{+\mathrm{i}k\Delta x}}\bigg|=\Big|\frac{\bar{A}}{A}\Big|\equiv1
\]
因此该差分格式是弱稳定的.
\end{solution} 

\begin{problem}[问题4]
分析对流方程的Warming-Beam差分格式的精度和稳定性.
\begin{eqnarray}
u_j^{\overline{n+1}} &=& u_j^n - r\big(u_j^n-u_{j-1}^n\big)\nonumber\\
u_j^{n+1} &=& \frac{1}{2}\big(u_j^n+u_j^{\overline{n+1}} \big)
-\frac{1}{2}r\Big[
\big(u_j^n-2u_{j-1}^n + u_{j-2}^n\big)
+ \big( u_j^{\overline{n+1}} - u_{j-1}^{\overline{n+1}}\big)
\Big]\nonumber
\end{eqnarray}
\end{problem}

\begin{solution}
\textbf{精度分析:} 将$u_{j}^{n+1}$, $u_{j-1}^{n}$, $u_{j-2}^{n}$分别在$u=u_{j}^{n}$展开成泰勒级数
{\setlength\arraycolsep{2pt}
\begin{eqnarray}
u_{j}^{n+1} & = & u_{j}^{n}+\frac{\partial u}{\partial t}\Delta t+\frac{1}{2}\frac{\partial^{2}u}{\partial t^{2}}\Delta t^{2}+\frac{1}{6}\frac{\partial^{3}u}{\partial t^{3}}\Delta t^{3}+O(\Delta t^{4})\label{eq:ujn+1}\\
u_{j-1}^{n} & = & u_{j}^{n}-\frac{\partial u}{\partial x}\Delta x+\frac{1}{2}\frac{\partial^{2}u}{\partial x^{2}}\Delta x^{2}-\frac{1}{6}\frac{\partial^{3}u}{\partial x^{3}}\Delta x^{3}+O(\Delta x^{4})\label{eq:ujn-1}\\
u_{j-2}^{n} & = & u_{j}^{n}-\frac{\partial u}{\partial x}2\Delta x+\frac{1}{2}\frac{\partial^{2}u}{\partial x^{2}}(2\Delta x)^{2}-\frac{1}{6}\frac{\partial^{3}u}{\partial x^{3}}(2\Delta x)^{3}+O(\Delta x^{4})\label{eq:uj-2n}
\end{eqnarray}}
将题中第一式代入第二式得
{\setlength\arraycolsep{2pt}
\begin{eqnarray}
u_{j}^{n+1} & = & \frac{1}{2}\Big[u_{j}^{n}+u_{j}^{n}-r\big(u_{j}^{n}-u_{j-1}^{n}\big)\Big]-\frac{1}{2}r\Big[\big(u_{j}^{n}-2u_{j-1}^{n}+u_{j-2}^{n}\big)+\nonumber\\
 &  & \hphantom{\frac{1}{2}\bigg[u_{j}^{n}+u_{j}^{n}-r}+u_{j}^{n}-r\big(u_{j}^{n}-u_{j-1}^{n}\big)-u_{j-1}^{n}+r\big(u_{j-1}^{n}-u_{j-2}^{n}\big)\Big]\nonumber\\
 & = & u_{j}^{n}-r(u_{j}^{n}-u_{j-1}^{n})-\frac{1}{2}r(1-r)\big(u_{j}^{n}-2u_{j-1}^{n}+u_{j-2}^{n}\big)\label{eq:ujn+1=}
\end{eqnarray}}
其中$r=a\frac{\Delta t}{\Delta x}$, 将$u_{j}^{n+1}$, $u_{j-1}^{n}$,
$u_{j-2}^{n}$的泰勒展开代入上式, 并注意到\textcolor{red}{$\frac{\partial^{2}u}{\partial t^{2}}-a^{2}\frac{\partial^{2}u}{\partial x^{2}}=0$}:
\begin{eqnarray*}
 &  & \frac{u_{j}^{n+1}-u_{j}^{n}}{\Delta t}+a\frac{u_{j}^{n}-u_{j-1}^{n}}{\Delta x}+\frac{a}{2}(\Delta x-a\Delta t)\frac{u_{j}^{n}-2u_{j-1}^{n}+u_{j-2}^{n}}{\Delta x\Delta x}\\
 & = & \textcolor{DarkGreen}{\frac{\partial u}{\partial t}}+\textcolor{red}{\frac{1}{2}\frac{\partial^{2}u}{\partial t^{2}}\Delta t}+\textcolor{DarkGreen}{\frac{1}{6}\frac{\partial^{3}u}{\partial t^{3}}\Delta t^{2}}+O(\Delta t^{3})+\textcolor{DarkGreen}{a\frac{\partial u}{\partial x}}\textcolor{magenta}{-\frac{a}{2}\frac{\partial^{2}u}{\partial x^{2}}\Delta x}+
\textcolor{blue}{\frac{a}{6}\frac{\partial^{3}u}{\partial x^{3}}\Delta x^{2}}
 +O(\Delta x^{3})+\\
 &  & \hphantom{\frac{\partial u}{\partial t}}
 +\frac{a}{2}\bigg(\textcolor{magenta}{\frac{\partial^{2}u}{\partial x^{2}}\Delta x}\textcolor{blue}{-\frac{\partial^{3}u}{\partial x^{3}}\Delta x^{2}}+O(\Delta x^{3})\bigg)
 -\frac{a^2}{2}\bigg(\textcolor{red}{\frac{\partial^{2}u}{\partial x^{2}}\Delta t}\textcolor{DarkGreen}{-\frac{\partial^{3}u}{\partial x^{3}}\Delta x\Delta t}+O(\Delta x^{2}\Delta t)\bigg)\\
 & = & \textcolor{DarkGreen}{\frac{\partial u}{\partial t}}+\textcolor{DarkGreen}{\frac{1}{6}\frac{\partial^{3}u}{\partial t^{3}}\Delta t^{2}}+\textcolor{DarkGreen}{a\frac{\partial u}{\partial x}}\textcolor{blue}{-\frac{a}{3}\frac{\partial^{3}u}{\partial x^{3}}\Delta x^{2}}+\textcolor{DarkGreen}{\frac{a^{2}}{2}\frac{\partial^{3}u}{\partial x^{3}}\Delta x\Delta t}+O(\Delta t^{3},\Delta x^{3},\Delta x^{2}\Delta t)\\
 & = & \frac{\partial u}{\partial t}+a\frac{\partial u}{\partial x}+O(\Delta t^{2},\Delta x^{2},\Delta t\Delta x)
\end{eqnarray*}
因此该差分格式在时间和空间上都是二阶精度.

\vspace{1em}
\noindent\textbf{稳定性分析:} 对差分方程中的各项作Fourier展开
$u_{j}^{n}=\sum_{k=1}^{\infty}\hat{u}_{j}^{n}\mathrm{e}^{\mathrm{i}jk\Delta x}$
%u_{j-1}^{n}=\sum_{k=1}^{\infty}\hat{u}_{j-1}^{n+1}\mathrm{e}^{\mathrm{i}(j-1)k\Delta x},\:
%u_{j-1\pm 1}^{n}=\sum_{k=1}^{\infty}\hat{u}_{j-1\pm1}^{n+1}\mathrm{e}^{\mathrm{i}(j-1\pm1)k\Delta x}
并代入式(\ref{eq:ujn+1=})得到它的每个分量的误差方程:
{\setlength\arraycolsep{2pt}
\begin{eqnarray*}
\hat{u}_{j}^{n+1}\mathrm{e}^{\mathrm{i}jk\Delta x} & = & \hat{u}_{j}^{n}\mathrm{e}^{\mathrm{i}jk\Delta x}-r(\hat{u}_{j}^{n}\mathrm{e}^{\mathrm{i}jk\Delta x}-\hat{u}_{j-1}^{n}\mathrm{e}^{\mathrm{i}(j-1)k\Delta x})\\
 &  & \hphantom{\hat{u}_{j}^{n}\mathrm{e}^{\mathrm{i}jk\Delta x}}-\frac{1}{2}r(1-r)\big(\hat{u}_{j}^{n}\mathrm{e}^{\mathrm{i}jk\Delta x}-2\hat{u}_{j-1}^{n}\mathrm{e}^{\mathrm{i}(j-1)k\Delta x}+\hat{u}_{j-2}^{n}\mathrm{e}^{\mathrm{i}(j-2)k\Delta x}\big)
\end{eqnarray*}}
放大因子为$G=\hat{u}_{j}^{n+1}/\hat{u}_{j}^{n}$, 则上式可化为
{\setlength\arraycolsep{2pt}
\begin{eqnarray*}
G & = & 1-r(1-\mathrm{e}^{-\mathrm{i}jk\Delta x})-\frac{1}{2}r(1-r)\big(1-2\mathrm{e}^{-\mathrm{i}jk\Delta x}+\mathrm{e}^{-2\mathrm{i}k\Delta x}\big)\\
 & = & 1-r(1-\mathrm{e}^{-\mathrm{i}jk\Delta x})-\frac{1}{2}r(1-r)\big(1-\mathrm{e}^{-\mathrm{i}jk\Delta x}\big)^{2}\\
 & = & 1-rz-\frac{1}{2}r(1-r)z^{2}
\end{eqnarray*}}
其中$z=1-\mathrm{e}^{-\mathrm{i}jk\Delta x}=1-\cos\theta+\mathrm{i}\sin\theta$.
注意到$z\bar{z}=z+\bar{z}=2(1-\cos\theta)=2s$, 其中$0\leq s\leq2$. 则有
{\setlength\arraycolsep{2pt}
\begin{eqnarray*}
|G|^{2} & = & G\bar{G}=\Big[1-rz-\frac{1}{2}r(1-r)z^{2}\Big]\Big[1-r\bar{z}-\frac{1}{2}r(1-r)\bar{z}^{2}\Big]\\
 & = & 1\textcolor{red}{-rz}\textcolor{blue}{-\frac{1}{2}r(1-r)z^{2}}\qquad
       \textcolor{red}{-r\bar{z}}+r^{2}z\bar{z}+\textcolor{DarkGreen}{\frac{1}{2}r^2(1-r)z^{2}\bar{z}}\\
 &  & \hphantom{1-rz-\frac{1}{2}r(1-r)z^{2}\qquad}
 \textcolor{blue}{-\frac{1}{2}r(1-r)\bar{z}^{2}}+\textcolor{DarkGreen}{\frac{1}{2}r^{2}(1-r)\bar{z}^{2}z}+\frac{1}{4}r^{2}(1-r)^{2}(z\bar{z})^{2}\\
 & = & 1\textcolor{red}{-r(z+\bar{z})}\textcolor{blue}{-\frac{1}{2}r(1-r)(z^{2}+\bar{z}^{2})}+r^{2}z\bar{z}+\textcolor{DarkGreen}{\frac{1}{2}r^{2}(1-r)z\bar{z}(z+\bar{z})}+\frac{1}{4}r^{2}(1-r)^{2}(z\bar{z})^{2}\\
 & = & 1-2rs-2r(1-r)\big(s^{2}-s\big)+2r^{2}s+2r^{2}(1-r)s^{2}+r^{2}(1-r)^{2}s^{2}\\
 & = & 1-2r(1-(1-r)-r)s+r(1-r)(-2+3r-r^{2})s^{2}\\
 & = & 1-r(1-r)^{2}(2-r)s^{2}
\end{eqnarray*}}
显然当$0<r=a\frac{\Delta t}{\Delta x}\leq2$时$|G|\leq1$, 此时差分格式是稳定的.
\end{solution} 

\begin{problem}[问题5]
分析对流方程的紧致差分格式的精度和稳定性:
\[
u_j^{n+1} = u_j^n -rF_j^n
\]
\[
-\frac{1}{3}F_{j+1}^n + \frac{2}{3}F_j^n + \frac{2}{3}F_{j-1}^n
= -\frac{1}{2}\big(u_{j+1}^n-u_j^n\big) + \frac{3}{2}\big(u_j^n-u_{j-1}^n\big)
\]
\end{problem}

\begin{solution}
\textbf{精度分析:} 将$u_{j+1}^{n+1}$, $u_{j-1}^{n+1}$, $u_{j+1}^{n}$, $u_{j}^{n+1}$,
$u_{j-1}^{n}$分别在$u_{j}^{n}$处展开成泰勒级数
{\setlength\arraycolsep{2pt}
\begin{eqnarray*}
\textcolor{red}{u_{j+1}^{n+1}} & = & u_{j}^{n}+\Big(\Delta t\frac{\partial}{\partial t}+\Delta x\frac{\partial}{\partial x}\Big)u+\frac{1}{2}\Big(\Delta t\frac{\partial}{\partial t}+\Delta x\frac{\partial}{\partial x}\Big)^{2}u+\frac{1}{6}\Big(\Delta t\frac{\partial}{\partial t}+\Delta x\frac{\partial}{\partial x}\Big)^{3}u+\cdots\\
\textcolor{blue}{u_{j-1}^{n+1}} & = & u_{j}^{n}+\Big(\Delta t\frac{\partial}{\partial t}-\Delta x\frac{\partial}{\partial x}\Big)u+\frac{1}{2}\Big(\Delta t\frac{\partial}{\partial t}-\Delta x\frac{\partial}{\partial x}\Big)^{2}u+\frac{1}{6}\Big(\Delta t\frac{\partial}{\partial t}-\Delta x\frac{\partial}{\partial x}\Big)^{3}u+\cdots\\
\textcolor{red}{u_{j+1}^{n\hphantom{+1}}} & = & u_{j}^{n}+\frac{\partial u}{\partial x}\Delta x+\frac{1}{2}\frac{\partial^{2}u}{\partial x^{2}}\Delta x^{2}+\frac{1}{6}\frac{\partial^{3}u}{\partial x^{3}}\Delta x^{3}+O(\Delta x^{4})\\
\textcolor{blue}{ u_{j-1}^{n\hphantom{+1}}} & = & u_{j}^{n}-\frac{\partial u}{\partial x}\Delta x+\frac{1}{2}\frac{\partial^{2}u}{\partial x^{2}}\Delta x^{2}-\frac{1}{6}\frac{\partial^{3}u}{\partial x^{3}}\Delta x^{3}+O(\Delta x^{4})\\
\textcolor[rgb]{0.33,0.33,0.33}{u_{j}^{n+1}} & = & u_{j}^{n}+\frac{\partial u}{\partial t}\Delta t\:+\frac{1}{2}\frac{\partial^{2}u}{\partial t^{2}}\Delta t^{2}\:+\frac{1}{6}\frac{\partial^{3}u}{\partial t^{3}}\Delta t^{3}\:+O(\Delta t^{4}\:)
\end{eqnarray*}}
因此有
{\setlength\arraycolsep{2pt}
\begin{eqnarray*}
\textcolor{red}{\frac{u_{j+1}^{n+1}-u_{j+1}^{n}}{\Delta t}}
          & \textcolor{red}{=} & \textcolor{red}{\frac{\partial u}{\partial t}+\frac{1}{2}\frac{\partial^{2}u}{\partial t^{2}}\Delta t+\frac{\partial^{2}u}{\partial x\partial t}\Delta x
          +\frac{1}{6}\frac{\partial^{3}u}{\partial t^{3}}\Delta t^{2}+\frac{1}{2}\frac{\partial^{3}u}{\partial t\partial x^{2}}\Delta x^{2}+\frac{1}{2}\frac{\partial^{3}u}{\partial t^{2}\partial x}\Delta t\Delta x+\cdots}\\
\textcolor{blue}{\frac{u_{j-1}^{n+1}-u_{j-1}^{n}}{\Delta t}}
          & \textcolor{blue}{=} & \textcolor{blue}{\frac{\partial u}{\partial t}+\frac{1}{2}\frac{\partial^{2}u}{\partial t^{2}}\Delta t-\frac{\partial^{2}u}{\partial x\partial t}\Delta x+
          \frac{1}{6}\frac{\partial^{3}u}{\partial t^{3}}\Delta t^{2}+\frac{1}{2}\frac{\partial^{3}u}{\partial t\partial x^{2}}\Delta x^{2}-\frac{1}{2}\frac{\partial^{3}u}{\partial t^{2}\partial x}\Delta t\Delta x+\cdots}\\
\textcolor[rgb]{0.33,0.33,0.33}{\frac{u_{j}^{n+1}-u_{j\hphantom{-1}}^{n}}{\Delta t}}
          & \textcolor[rgb]{0.33,0.33,0.33}{=} & \textcolor[rgb]{0.33,0.33,0.33}{\frac{\partial u}{\partial t}+\frac{1}{2}\frac{\partial^{2}u}{\partial t^{2}}\Delta t\:+\frac{1}{6}\frac{\partial^{3}u}{\partial t^{3}}\Delta t^{2}\:+\cdots}\\
\textcolor{DarkGreen}{\frac{u_{j+1}^{n\hphantom{+1}}-u_{j\hphantom{-1}}^{n}}{\Delta x}}
          & \textcolor{DarkGreen}{=} & \textcolor{DarkGreen}{\frac{\partial u}{\partial x}+\frac{1}{2}\frac{\partial^{2}u}{\partial x^{2}}\Delta x+\frac{1}{6}\frac{\partial^{3}u}{\partial x^{3}}\Delta x^{2}+\cdots}\\
\textcolor[rgb]{0.6,0.3,0.1}{\frac{u_{j}^{n\hphantom{-1}}-u_{j-1}^{n}}{\Delta x}}
          & \textcolor[rgb]{0.6,0.3,0.1}{=} & \textcolor[rgb]{0.6,0.3,0.1}{\frac{\partial u}{\partial x}-\frac{1}{2}\frac{\partial^{2}u}{\partial x^{2}}\Delta x+\frac{1}{6}\frac{\partial^{3}u}{\partial x^{3}}\Delta x^{2}+\cdots}
\end{eqnarray*}}
将题中的差分格式合并成如下差分方程
\[
\frac{1}{3}\textcolor{red}{\frac{u_{j+1}^{n+1}-u_{j+1}^{n}}{\Delta t}}
-\frac{2}{3}\textcolor[rgb]{0.33,0.33,0.33}{\frac{u_{j}^{n+1}-u_{j}^{n}}{\Delta t}}
-\frac{2}{3}\textcolor{blue}{\frac{u_{j-1}^{n+1}-u_{j-1}^{n}}{\Delta t}}
=
-\frac{1}{2}a\textcolor{DarkGreen}{\frac{u_{j+1}^{n}-u_{j}^{n}}{\Delta x}}
+\frac{3}{2}a\textcolor[rgb]{0.6,0.3,0.1}{\frac{u_{j}^{n}-u_{j-1}^{n}}{\Delta x}}
\]
上式的左边和右边分别如下(仅保留了二阶项)
\begin{eqnarray*}
\text{左边} & = & +\frac{1}{3}\textcolor{red}{\Bigg[\hphantom{-}\frac{\partial u}{\partial t}+\frac{1}{2}\frac{\partial^{2}u}{\partial t^{2}}\Delta t+\frac{\partial^{2}u}{\partial x\partial t}\Delta x+\frac{1}{6}\frac{\partial^{3}u}{\partial t^{3}}\Delta t^{2}+\frac{1}{2}\frac{\partial^{3}u}{\partial t\partial x^{2}}\Delta x^{2}+\frac{1}{2}\frac{\partial^{3}u}{\partial t^{2}\partial x}\Delta x\Delta t\Bigg]}\\
 &  & -\frac{2}{3}\textcolor[rgb]{0.33,0.33,0.33}{\Bigg[\hphantom{-}\frac{\partial u}{\partial t}+\frac{1}{2}\frac{\partial^{2}u}{\partial t^{2}}\Delta t\hphantom{\,\,+\frac{\partial^{2}u}{\partial x\partial t}\Delta x\,}+\frac{1}{6}\frac{\partial^{3}u}{\partial t^{3}}\Delta t^{2}\hphantom{+\frac{1}{2}\frac{\partial^{3}u}{\partial t\partial x^{2}}\Delta x^{2}+\frac{1}{2}\frac{\partial^{3}u}{\partial t^{2}\partial x}\Delta x\Delta t\;}\Bigg]}\\
 &  & -\frac{2}{3}\textcolor{blue}{\Bigg[\hphantom{-}\frac{\partial u}{\partial t}+\frac{1}{2}\frac{\partial^{2}u}{\partial t^{2}}\Delta t-\frac{\partial^{2}u}{\partial x\partial t}\Delta x+\frac{1}{6}\frac{\partial^{3}u}{\partial t^{3}}\Delta t^{2}+\frac{1}{2}\frac{\partial^{3}u}{\partial t\partial x^{2}}\Delta x^{2}-\frac{1}{2}\frac{\partial^{3}u}{\partial t^{2}\partial x}\Delta x\Delta t\Bigg]}\\
 & = & \hphantom{-\frac{2}{3}\Bigg[}\!\!-\frac{\partial u}{\partial t}-\frac{1}{2}\frac{\partial^{2}u}{\partial t^{2}}\Delta t+\frac{\partial^{2}u}{\partial x\partial t}\Delta x-\frac{1}{6}\frac{\partial^{3}u}{\partial t^{3}}\Delta t^{2}-\frac{1}{6}\frac{\partial^{3}u}{\partial t\partial x^{2}}\Delta x^{2}+\frac{1}{2}\frac{\partial^{3}u}{\partial t^{2}\partial x}\Delta x\Delta t\hphantom{\Bigg]}\\
\\
\text{右边} & = & -\frac{1}{2}a\textcolor{DarkGreen}{\Bigg[\frac{\partial u}{\partial x}+\frac{1}{2}\frac{\partial^{2}u}{\partial x^{2}}\Delta x+\frac{1}{6}\frac{\partial^{3}u}{\partial x^{3}}\Delta x^{2}\Bigg]} +\frac{3}{2}a\textcolor[rgb]{0.6,0.3,0.1}{\Bigg[\frac{\partial u}{\partial x}-\frac{1}{2}\frac{\partial^{2}u}{\partial x^{2}}\Delta x+\frac{1}{6}\frac{\partial^{3}u}{\partial x^{3}}\Delta x^{2}\Bigg]}\\
 & = & a\frac{\partial u}{\partial x}-a\frac{\partial^{2}u}{\partial x^{2}}\Delta x+a\frac{1}{6}\frac{\partial^{3}u}{\partial x^{3}}\Delta x^{2}
\end{eqnarray*}
因此有
\begin{align*}
\text{右边-左边} & =a\frac{\partial u}{\partial x}-a\frac{\partial^{2}u}{\partial x^{2}}\Delta x+\frac{1}{6}a\frac{\partial^{3}u}{\partial x^{3}}\Delta x^{2}+\frac{\partial u}{\partial t}-\frac{\partial^{2}u}{\partial x\partial t}\Delta x+\frac{1}{6}\frac{\partial^{3}u}{\partial t\partial x^{2}}\Delta x^{2}+O(\Delta t,\Delta x\Delta t)\\
 & =\frac{\partial u}{\partial t}+a\frac{\partial u}{\partial x}-\Big(a\frac{\partial^{2}u}{\partial x^{2}}+\frac{\partial^{2}u}{\partial x\partial t}\Big)\Delta x+\frac{1}{6}\Big(a\frac{\partial^{3}u}{\partial x^{3}}+\frac{\partial^{3}u}{\partial t\partial x^{2}}\Big)\Delta x^{2}+O(\Delta t,\Delta x\Delta t)\\
 & =\frac{\partial u}{\partial t}+a\frac{\partial u}{\partial x}+O(\Delta t,\Delta x\Delta t)
\end{align*}
因此该差分格式在时间上是一阶精度, 空间上二阶精度.

\vspace{1em}
\noindent\textbf{稳定性分析:} 对差分方程中的各项作Fourier展开
$u_{j}^{n}=\sum_{k=1}^{\infty}\hat{u}_{j}^{n}\mathrm{e}^{\mathrm{i}jk\Delta x}$
并代入式差分方程得到它的每个分量的误差方程:
{\setlength\arraycolsep{2pt}
\begin{eqnarray*}
\frac{1}{3}(G-1)\mathrm{e}^{\mathrm{i}k\Delta x}-\frac{2}{3}(G-1)-\frac{2}{3}(G-1)\mathrm{e}^{-\mathrm{i}k\Delta x} & = & -\frac{1}{2}r(\mathrm{e}^{\mathrm{i}k\Delta x}-1)+\frac{3}{2}r(1-\mathrm{e}^{-\mathrm{i}k\Delta x})\\
 & \Downarrow\\
2(G-1)\Big(\mathrm{e}^{\mathrm{i}k\Delta x}-2-2\mathrm{e}^{-\mathrm{i}k\Delta x}\Big) & = & 3r(1-\mathrm{e}^{\mathrm{i}k\Delta x})+9r(1-\mathrm{e}^{-\mathrm{i}k\Delta x})\\
 & \Downarrow\\
2(G-1)\Big[-\Big(1-\mathrm{e}^{\mathrm{i}k\Delta x}\Big)+2\Big(1-\mathrm{e}^{-\mathrm{i}k\Delta x}\Big)-3\Big] & = & 3r(1-\mathrm{e}^{\mathrm{i}k\Delta x})+9r(1-\mathrm{e}^{-\mathrm{i}k\Delta x})
\end{eqnarray*}}
其中$G=\hat{u}_{j}^{n+1}/\hat{u}_{j}^{n}$为放大因子. 令$1-\mathrm{e}^{-\mathrm{i}k\Delta x}=z$, 则上式可化为
\[
G=\frac{(3r-2)z+(9r+4)\bar{z}-6}{2(2\bar{z}-z-3)}=\frac{az+b\bar{z}-6}{4\bar{z}-2z-6}
\]
其中$\bar{z}$为$z$的复共轭, $a=3r-2$, $b=9r+4$. 令$z\bar{z}=z+\bar{z}=2(1-\cos\theta)=s$($0\leq s\leq 4$)则有
{\setlength\arraycolsep{2pt}
\begin{eqnarray*}
|G|^2&=&G\bar{G} = \frac{(az+b\bar{z}-6)(a\bar{z}+bz-6)}{(4\bar{z}-2z-6)(4z-2\bar{z}-6)}\\
     &=&\frac{(a^{2}+b^{2})z\hat{z}+ab(\hat{z}+z)^{2}-2abz\hat{z}-6(a+b)(\hat{z}+z)+36}
     {36z\hat{z}-8(\hat{z}+z)^{2}-12(\hat{z}+z)+36}\\
     &=& \frac{(a-b)^2s+abs^{2}-6(a+b)s+36}{24s-8s^{2}+36}\\
\end{eqnarray*}}
显然上式分母大于0. 为比较分子与分母的大小, 将分子分母作差:
\begin{eqnarray*}
\text{分子-分母}&=&\Big((a-b)^2s+abs^{2}-6(a+b)s+36\Big) - (24s-8s^{2}+36)\\
                &=&(a-b)^2s+abs^{2}-6(a+b)s- 24s+8s^{2}\\
                &=&s\Big((a-b)^2+(ab+8)s-6(a+b)- 24\Big)\\
                &=&s\Big(36(r+1)^2+(27r^2-6r)s-12(6r+1)- 24\Big)\\
                &=&s\Big(36r^2+72r+36+27r^2s-6rs-72r-12- 24\Big)\\
                &=&3sr(12r+9rs-2s)
\end{eqnarray*}
因此$|G|\leq 1$成立的条件为
\[
12r+9rs-2s<0\:\Rightarrow\: r<\frac{2s}{12+9s} = \frac{2}{12/s+9}
\]
由于$0\leq s\leq 4$, 因此上述条件在$r>0$时不一定满足, 因此该差分格式是不稳定的.
%即$0<r=a\frac{\Delta t}{\Delta x}\leq 1/6$时, 该差分格式是稳定的.
\end{solution} 

\begin{problem}[问题6]
分析热传导方程$\frac{\partial u}{\partial t} - \beta\frac{\partial^2 u}{\partial x^2} = 0$, $\beta>0$如下差分格式($\sigma = \frac{\beta\Delta t}{\Delta x^2}$)的稳定性:
\[
u_j^{n+1} = u_j^{n-1} + \frac{2}{3}\sigma
\big[
\delta_x^2 u_j^{n+1} + \delta_x^2 u_j^n + \delta_x^2 u_j^{n-1}
\big]
\]
其中$\delta_x^2 u_j^n = (u_{j+1}^n - 2u_j^n + u_{j-1}^n)$.
\end{problem}

\begin{solution}
将$\delta_{x}^{2}u_{j}^{n}=(u_{j+1}^{n}-2u_{j}^{n}+u_{j-1}^{n})$代入原题中的差分方程得
\[
u_{j}^{n+1}=u_{j}^{n-1}+\frac{2}{3}\sigma\sum_{k=n-1}^{n+1}(u_{j+1}^{k}-2u_{j}^{k}+u_{j-1}^{k})
%\big[(u_{j+1}^{n+1}-2u_{j}^{n+1}+u_{j-1}^{n+1})+(u_{j+1}^{n}-2u_{j}^{n}+u_{j-1}^{n})+(u_{j+1}^{n-1}-2u_{j}^{n-1}+u_{j-1}^{n-1})\big]
\]
对差分方程中的各项作Fourier展开$u_{j}^{n}=\sum_{k=1}^{\infty}\hat{u}_{j}^{n}\mathrm{e}^{\mathrm{i}jk\Delta x}$
并代入上式得到它的每个分量的误差方程:
{\setlength\arraycolsep{2pt}
\begin{eqnarray*}
\hat{u}_{j}^{n+1} & = & \hat{u}_{j}^{n-1}+\frac{2}{3}\sigma(\mathrm{e}^{\mathrm{i}k\Delta x}-2+\mathrm{e}^{-\mathrm{i}k\Delta x})\big[\hat{u}_{j}^{n+1}+\hat{u}_{j}^{n}+\hat{u}_{j}^{n-1}\big]\\
 & = & \hat{u}_{j}^{n-1}+\frac{4}{3}\sigma(\cos\theta-1)\big[\hat{u}_{j}^{n+1}+\hat{u}_{j}^{n}+\hat{u}_{j}^{n-1}\big]\\
 & = & A\hat{u}_{j}^{n+1}+A\hat{u}_{j}^{n}+(1+A)\hat{u}_{j}^{n-1}
\end{eqnarray*}}
其中$A=\frac{4}{3}\sigma(\cos\theta-1)$. 这是一个三层差分方程, 为此, 引入新变量$v$,
并令$v_{j}^{n+1}=u_{j}^{n}$, $\mathbf{u}=[u,\, v]^{T}$, 则上式可化为
\[
\hat{\mathbf{u}}_{j}^{n+1}=\left[\begin{array}{c}
\hat{u}_{j}^{n+1}\\
\hat{v}_{j}^{n+1}
\end{array}\right]=\left[\begin{array}{cc}
\frac{A}{1-A} & \frac{1+A}{1-A}\\
1 & 0
\end{array}\right]\left[\begin{array}{c}
\hat{u}_{j}^{n}\\
\hat{v}_{j}^{n}
\end{array}\right]
\]
特征方程为
\[
\left|\begin{array}{cc}
\frac{A}{1-A}-\lambda & \frac{1+A}{1-A}\\
1 & -\lambda
\end{array}\right|=\lambda^{2}+\frac{A}{A-1}\lambda+\frac{A+1}{A-1}=0
\]
特征根为
\begin{align*}
\lambda_{1,2} & =\frac{-\frac{A}{A-1}\pm\sqrt{\frac{A^{2}}{(A-1)^{2}}-\frac{4(A+1)(A-1)}{(A-1)^{2}}}}{2}=\frac{A\pm\sqrt{A^{2}-4(A^{2}-1)}}{2(1-A)}\\
 & =\frac{A\pm\sqrt{4-3A^{2}}}{2(1-A)}=\frac{4\sigma(\cos\theta-1)\pm2\sqrt{9-12\sigma^{2}(\cos\theta-1)^{2}}}{6-8\sigma(\cos\theta-1)}
\end{align*}
下面分两种情况讨论
\begin{itemize}
\item 当$9-12\sigma^{2}(\cos\theta-1)^{2}\geq0$时. 特征方程有两个实根. 有绝对值最大值:
\begin{align*}
|\lambda_{\max}| & =\Bigg|\frac{4\sigma(\cos\theta-1)-2\sqrt{9-12\sigma^{2}(\cos\theta-1)^{2}}}{6-8\sigma(\cos\theta-1)}\Bigg|\\
 & \leq\Bigg|\frac{4\sigma(1-\cos\theta)+2\sqrt{9}}{6+8\sigma(1-\cos\theta)}\Bigg|=\frac{4\sigma(1-\cos\theta)+6}{8\sigma(1-\cos\theta)+6}\\
 & \leq1
\end{align*}
 此时差分格式恒稳定.
\item 当$9-12\sigma^{2}(\cos\theta-1)^{2}<0$.特征方程有两个复根.有绝对值最大值:
\begin{align*}
|\lambda_{\max}| & =\Bigg[\frac{16\sigma^{2}(\cos\theta-1)^{2}-36+48\sigma^{2}(\cos\theta-1)^{2}}{\big(6+8\sigma(1-\cos\theta)\big)^{2}}\Bigg]^{1/2}\\
 & =\Bigg[\frac{60\sigma^{2}(1-\cos\theta)^{2}-36}{\big(6+8\sigma(1-\cos\theta)\big)^{2}}\Bigg]^{1/2}\leq\Bigg[\frac{64\sigma^{2}(1-\cos\theta)^{2}-36}{\big(6+8\sigma(1-\cos\theta)\big)^{2}}\Bigg]^{1/2}\\
 & \leq\sqrt{\frac{8\sigma(1-\cos\theta)-6}{8\sigma(1-\cos\theta)+6}}\\
 & <1
\end{align*}
 此时差分格式恒稳定.
\end{itemize}
综上所述, 对于任何$\sigma=\beta\Delta t/\Delta x^{2}>0$, 该差分格式恒稳定.
\end{solution}

\begin{problem}[问题7]
分析波动方程$\frac{\partial^2 u}{\partial t^2}-a^2\frac{\partial^2u}{\partial x^2}=0$, $a>0$下列差分格式($r=\frac{a\Delta t}{\Delta x}$)的稳定性:
\begin{eqnarray}
u_j^{n+1} &=& \frac{1}{2}\big(u_{j+1}^n+u_{j-1}^n\big) - \frac{1}{2}r\big(v_{j+1}^n-v_{j-1}^n\big)\nonumber\\
v_j^{n+1} &=& \frac{1}{2}\big(v_{j+1}^n+v_{j-1}^n\big) - \frac{1}{2}r\big(u_{j+1}^n-u_{j-1}^n\big)\nonumber
\end{eqnarray}
\end{problem}

\begin{solution}
对差分方程中的各项作Fourier展开$u_{j}^{n}=\sum_{k=1}^{\infty}\hat{u}_{j}^{n}\mathrm{e}^{\mathrm{i}jk\Delta x}$
并代入上式得到它的每个分量误差方程的矩阵形式:
\[
\left[\begin{array}{c}
\hat{u}_{j}^{n+1}\\
\hat{v}_{j}^{n+1}
\end{array}\right]=\left[\begin{array}{cc}
\hphantom{-}\frac{1}{2}\hphantom{r}\Big(\mathrm{e}^{\mathrm{i}k\Delta x}+\mathrm{e}^{-\mathrm{i}k\Delta x}\Big) & -\frac{1}{2}r\Big(\mathrm{e}^{\mathrm{i}k\Delta x}-\mathrm{e}^{-\mathrm{i}k\Delta x}\Big)\\
-\frac{1}{2}r\Big(\mathrm{e}^{\mathrm{i}k\Delta x}-\mathrm{e}^{-\mathrm{i}k\Delta x}\Big) & \hphantom{-}\frac{1}{2}\hphantom{r}\Big(\mathrm{e}^{\mathrm{i}k\Delta x}+\mathrm{e}^{-\mathrm{i}k\Delta x}\Big)
\end{array}\right]\left[\begin{array}{c}
\hat{u}_{j}^{n}\\
\hat{v}_{j}^{n}
\end{array}\right]
\]
上述方程的特征方程为
\[
\left|\begin{array}{cc}
\hphantom{-}\frac{1}{2}\hphantom{r}\Big(\mathrm{e}^{\mathrm{i}k\Delta x}+\mathrm{e}^{-\mathrm{i}k\Delta x}\Big)-\lambda & -\frac{1}{2}r\Big(\mathrm{e}^{\mathrm{i}k\Delta x}-\mathrm{e}^{-\mathrm{i}k\Delta x}\Big)\hphantom{-\lambda}\\
-\frac{1}{2}r\Big(\mathrm{e}^{\mathrm{i}k\Delta x}-\mathrm{e}^{-\mathrm{i}k\Delta x}\Big)\hphantom{-\lambda} & \hphantom{-}\frac{1}{2}\hphantom{r}\Big(\mathrm{e}^{\mathrm{i}k\Delta x}+\mathrm{e}^{-\mathrm{i}k\Delta x}\Big)-\lambda
\end{array}\right|=0
\]
即
\[
\big[\cos\theta-\lambda\big]^{2}-\frac{1}{4}r^{2}\big(2\mathrm{i}\sin\theta\big)^{2}=\lambda^{2}-2\lambda\cos\theta+r^{2}\sin^{2}\theta+\cos^{2}\theta=0
\]
特征根为
\begin{align*}
\lambda_{1,2} & =\frac{2\cos\theta\pm\sqrt{4\cos^{2}\theta-4(r^{2}\sin^{2}\theta+\cos^{2}\theta)}}{2}=\cos\theta\pm\mathrm{i}\sin\theta r
\end{align*}
绝对值最大值为
\[
|\lambda_{\max}|=\cos^{2}\theta+r\sin^{2}\theta=1+(r-1)\sin^{2}\theta
\]
显然在$0<r<1$的条件下, $|\lambda_{\max}|\leq1$成立, 此时该差分格式是稳定的.
\end{solution} 

\begin{problem}[问题8]
用 Taylor 分析法求出对流方程的L-W差分格式:
\[
u_j^{n+1} = u_j^n - \frac{1}{2}r\big(u_{j+1}^n - u_{j-1}^n\big) +\frac{1}{2}r^2
\big(u_{j+1}^n - 2u_j^n + u_{j-1}^n\big)
\]
的修正方程, 并求出它的耗散项和色散项的表达式.
\end{problem}

\vspace{1em}
\noindent 将$u_{j}^{n+1}$, $u_{j-1}^{n}$, $u_{j-2}^{n}$分别在$u=u_{j}^{n}$展开成泰勒级数
{\setlength\arraycolsep{2pt}
\begin{eqnarray*}
u_{j}^{n+1} & = & u_{j}^{n}+\frac{\partial u}{\partial t}\Delta t\:+\frac{1}{2}\frac{\partial^{2}u}{\partial t^{2}}\Delta t^{2}\:+\frac{1}{6}\frac{\partial^{3}u}{\partial t^{3}}\Delta t^{3}\:+\frac{1}{24}\frac{\partial^{4}u}{\partial t^{4}}\Delta t^{4}\:+O(\Delta t^{5}\:)\\
u_{j+1}^{n\hphantom{+1}} & = & u_{j}^{n}+\frac{\partial u}{\partial x}\Delta x+\frac{1}{2}\frac{\partial^{2}u}{\partial x^{2}}\Delta x^{2}+\frac{1}{6}\frac{\partial^{3}u}{\partial x^{3}}\Delta x^{3}+\frac{1}{24}\frac{\partial^{4}u}{\partial x^{4}}\Delta x^{4}+O(\Delta x^{5})\\
u_{j-1}^{n\hphantom{+1}} & = & u_{j}^{n}-\frac{\partial u}{\partial x}\Delta x+\frac{1}{2}\frac{\partial^{2}u}{\partial x^{2}}\Delta x{}^{2}-\frac{1}{6}\frac{\partial^{3}u}{\partial x^{3}}\Delta x{}^{3}+\frac{1}{24}\frac{\partial^{4}u}{\partial x^{4}}\Delta x^{4}+O(\Delta x^{5})
\end{eqnarray*}}

\begin{landscape}
\begin{solution}
\noindent 将以上泰勒级数代入差分方程
\[
\frac{\partial u}{\partial t}\Delta t+\frac{1}{2}\frac{\partial^{2}u}{\partial t^{2}}\Delta t^{2}+\frac{1}{6}\frac{\partial^{3}u}{\partial t^{3}}\Delta t^{3}+\frac{1}{24}\frac{\partial^{4}u}{\partial t^{4}}\Delta t^{4}+O(\Delta t^{5})=-\frac{1}{2}r\Big[2\frac{\partial u}{\partial x}\Delta x+\frac{1}{3}\frac{\partial^{3}u}{\partial x^{3}}\Delta x^{3}+O(\Delta x^{5})\Big]+\frac{1}{2}r^{2}\Big[\frac{\partial^{2}u}{\partial x^{2}}\Delta x^{2}+\frac{1}{12}\frac{\partial^{4}u}{\partial x^{4}}\Delta x^{4}+O(\Delta x^{6})\Big]
\]
将$r=a\Delta t/\Delta x$代入上式并整理得
\[
0=u_{t}+au_{x}+\frac{1}{2}u_{tt}\Delta t-\frac{1}{2}a^{2}u_{xx}\Delta t+\frac{1}{6}u_{ttt}\Delta t^{2}+\frac{1}{6}au_{xxx}\Delta x^{2}+\frac{1}{24}u_{tttt}\Delta t^{3}-\frac{1}{24}a^{2}u_{xxxx}\Delta x^{2}\Delta t+\cdots
\]
采用图表法消元, 消元后可得到下表.
\begin{table}[!htb]
\centering
\caption{\label{tab:08}图表消元法列表}
\renewcommand\arraystretch{1.9}
{\scriptsize
\begin{tabular}{|c|r|c|c|c|c|c|c|c|c|c|c|c|c|c|c|}
\hline
(a) &  & $u_{t}$ & $u_{x}$ & $u_{tt}$ & $u_{xt}$ & $u_{xx}$ & $u_{ttt}$ & $u_{xtt}$ & $u_{xxt}$ & $u_{xxx}$ & $u_{tttt}$ & $u_{xttt}$ & $u_{xxtt}$ & $u_{xxxt}$ & $u_{xxxx}$\tabularnewline
\hline
\hline
(b) &  & \textcolor{black}{$1$} & \textcolor{red}{$a$} & \textcolor{DarkGreen}{$\frac{\Delta t}{2}$} &  & \textcolor{blue}{$-\frac{a^{2}}{2}\Delta t$} & \textcolor[rgb]{0.6,0.3,0.1}{$\frac{\Delta t^{2}}{6}$} &  &  & \textcolor{magenta}{$\frac{a}{6}\Delta x^{2}$} & \textcolor[rgb]{0.33,0.33,0.33}{$\frac{1}{24}\Delta t^{3}$} &  &  &  & \textcolor[rgb]{0.33,0.33,0.33}{$-\frac{a^{2}}{24}\Delta x^{2}\Delta t$}\tabularnewline
\hline
(c) & $-\frac{\Delta t}{2}\frac{\partial}{\partial t}\text{(b)}$ &  &  & \textcolor{black}{$-\frac{\Delta t}{2}$} & \textcolor{red}{$-\frac{a}{2}\Delta t$} &  & \textcolor{DarkGreen}{$-\frac{\Delta t^{2}}{4}$} &  & \textcolor{blue}{$\frac{a^{2}}{4}\Delta t^{2}$} &  & \textcolor[rgb]{0.6,0.3,0.1}{$-\frac{1}{12}\Delta t^{3}$} &  &  & \textcolor{magenta}{$-\frac{a}{12}\Delta x^{2}\Delta t$} & \tabularnewline
\hline
(d) & $a\frac{\Delta t}{2}\frac{\partial}{\partial x}\text{(b)}$ &  &  &  & \textcolor{black}{$\frac{a}{2}\Delta t$} & \textcolor{red}{$\frac{a^{2}}{2}\Delta t$} &  & $\frac{a}{4}\Delta t^{2}$ &  & \textcolor{blue}{$-\frac{a^{3}}{4}\Delta t^{2}$} &  & \textcolor[rgb]{0.6,0.3,0.1}{$\frac{a}{12}\Delta t^{3}$} &  &  & \textcolor{magenta}{$\frac{a^{2}}{12}\Delta x^{2}\Delta t$}\tabularnewline
\hline
(e) & $\frac{\Delta t^{2}}{12}\frac{\partial^{2}}{\partial t^{2}}\text{(b)}$ &  &  &  &  &  & \textcolor{black}{$\frac{\Delta t^{2}}{12}$} & \textcolor{red}{$\frac{a}{12}\Delta t^{2}$} &  &  & \textcolor{DarkGreen}{$\frac{1}{24}\Delta t^{3}$} &  & \textcolor{blue}{$-\frac{a^{2}}{24}\Delta t^{3}$} &  & \tabularnewline
\hline
(f) & $-a\frac{\Delta t^{2}}{3}\frac{\partial^{2}}{\partial t\partial x}\text{(b)}$ &  &  &  &  &  &  & \textcolor{black}{$-\frac{a}{3}\Delta t^{2}$} & \textcolor{red}{$-a^{2}\frac{\Delta t^{2}}{3}$} &  &  & \textcolor{DarkGreen}{$-\frac{a}{6}\Delta t^{3}$} &  & \textcolor{blue}{$\frac{a^{3}}{6}\Delta t^{3}$} & \tabularnewline
\hline
(g) & $a^{2}\frac{\Delta t^{2}}{12}\frac{\partial^{2}}{\partial x^{2}}\text{(b)}$ &  &  &  &  &  &  &  & \textcolor{black}{$a^{2}\frac{\Delta t^{2}}{12}$} & \textcolor{red}{$\frac{a^{3}}{12}\Delta t^{2}$} &  &  & \textcolor{DarkGreen}{$\frac{a^{2}}{24}\Delta t^{3}$} &  & \textcolor{blue}{$-\frac{a^{4}}{24}\Delta t^{3}$}\tabularnewline
\hline
(h) & $\frac{a}{12}\Delta t^{3}\frac{\partial^{3}}{\partial t^{2}\partial x}\text{(b)}$ &  &  &  &  &  &  &  &  &  &  & \textcolor{black}{$\frac{a}{12}\Delta t^{3}$} & \textcolor{red}{$\frac{a^{2}}{12}\Delta t^{3}$} &  & \tabularnewline
\hline
(i) & $-\frac{a^{2}}{12}\Delta t^{3}\frac{\partial^{3}}{\partial x^{2}\partial t}\text{(b)}$ &  &  &  &  &  &  &  &  &  &  &  & \textcolor{black}{$-\frac{a^{2}}{12}\Delta t^{3}$} & \textcolor{red}{$-\frac{a^{3}}{12}\Delta t^{3}$} & \tabularnewline
\hline
(j) & $(\frac{a}{12}\Delta x^{2}\Delta t-\frac{a^{3}}{12}\Delta t^{3})\frac{\partial^{3}}{\partial x^{3}}\text{(b)}$ &  &  &  &  &  &  &  &  &  &  &  &  & \textcolor{black}{$\frac{a}{12}\Delta x^{2}\Delta t-\frac{a^{3}}{12}\Delta t^{3}$} & \textcolor{red}{$\frac{a^{2}}{12}\Delta x^{2}\Delta t-\frac{a^{4}}{12}\Delta t^{3}$}\tabularnewline
\hline
\hline
 &  &  &  &  &  &  &  &  &  & $\mathbf{\frac{a}{6}\Delta x^{2}-a^{3}\frac{\Delta t^{2}}{6}}$ &  &  &  &  & $\mathbf{\frac{a^{2}}{8}\Delta x^{2}\Delta t-\frac{a^{4}}{8}\Delta t^{3}}$\tabularnewline
\hline
\end{tabular}}
\end{table}

\noindent 将表\ref{tab:08}中的(b)至(h)相加, 得到差分方程的修正方程(\textcolor{red}{色散项}和\textcolor{blue}{耗散项}已标出):
\[
u_{t}+au_{x}=-\big(\frac{a}{6}\Delta x^{2}-a^{3}\frac{\Delta t^{2}}{6}\big)u_{xxx}-\Big(\frac{a^{2}}{8}\Delta x^{2}\Delta t-\frac{a^{4}}{8}\Delta t^{3}\Big)u_{xxxx}+\cdots
=
\textcolor{red}{\underbrace{\frac{a\Delta x^{2}}{6}\big(r^{2}-1\big)u_{xxx}}_{\text{色散项}}}+\textcolor{blue}{\underbrace{\frac{a\Delta x^{3}}{8}r\Big(r^{2}-1\Big)u_{xxxx}}_{\text{耗散项}}}+\cdots
\]
\end{solution} 
\end{landscape}



\newpage
\appendix
\appendixpage
\lhead{2012年考试重点}                                                 %
\problemlist{\bf 计算流体力学2012年考试重点}

%\section{基础概念题}
\noindent 01. 试分析一维非定常等熵流方程组$\displaystyle\begin{cases}
\displaystyle\frac{\partial\rho}{\partial t}+u\frac{\partial\rho}{\partial x}+\rho\frac{\partial u}{\partial x}=0\\
\displaystyle\frac{\partial u}{\partial t}+u\frac{\partial u}{\partial x}+\frac{a^{2}}{\rho}\frac{\partial\rho}{\partial x}=0
\end{cases}$ 数学性质和类型.

\noindent 令$\mathbf{z}=\left[
                \begin{array}{c}
                  \rho \\
                  u \\
                \end{array}
              \right]$
则一维非定常等熵流方程组可写成
\[
\frac{\partial \mathbf{z}}{\partial t}
+
\left[
                                          \begin{array}{cc}
                                            u & \rho \\
                                            a^2/\rho & u \\
                                          \end{array}
                                        \right]
\frac{\partial \mathbf{z}}{\partial x} = 0
\]
 系数行列式特征方程为
\[
 \left|
\begin{array}{cc}
u-\lambda & \rho \\
 a^2/\rho & u-\lambda \\
 \end{array}
\right| = 0 \rightarrow
(u-\lambda)^2 - a^2 = 0 \rightarrow \lambda=u\pm a
 \]
为双曲型方程, 特征值为$u\pm a$.
 \vspace{1em}

\noindent 02. 对流方程, 热传导方程和对流-扩散方程的主要数学物理性质.\\
对流方程的数学物理性质:
\begin{itemize}
\item 它是双曲方程. 反映的流动特性可以用特征线来描述, 流场中扰动沿特征线传播, 在特征线上满足一定相容关系.
\item 它反映了流动中扰动传播的速度是有限的. 对于线性方程, 当$a>0$时, 扰动沿正方向; 当$a>0$时, 沿负方向.
\item 扰动波幅不衰减, 波动不变化. 无耗散和色散.
\item 通解存在依赖区和影响区. 扰动传播范围有限, 不影响全流场.
\end{itemize}
扩散方程(热传导方程)的数学物理性质:
\begin{itemize}
\item 它是抛物型方程, 有耗散无色散. 描述具有耗散机制的流动现象.
\item 任何局部扰动都会影响全流场.
\item 通解依赖于初始时刻粒子密度分布$\phi(\xi)$, 只要初值在$|x|<\infty$内连续有界, 则通解唯一, 连续的存在.
\item 它存在极值原理, 即如果初值是有界的, 且满足$m\leq \phi(x)\leq M$, 则通解也一定有界, 且满足$m\leq u(x,t)\leq M$
\end{itemize}
对流-扩散方程的数学物理性质:
\begin{itemize}
\item 它是双曲-抛物型方程. 解单值连续, 且永远存在.
\item 具有波动特性, 扰动沿特征线传播, 且传播速度是有限的.
\item 具有黏性流动特性, 有耗散无色散.
\end{itemize}
\vspace{1em}

\noindent 03. 什么是差分格式相容性, 收敛性, 稳定性
\begin{itemize}
\item 相容性: 对于足够光滑的函数, 若$\Delta x\rightarrow 0$, $\Delta t\rightarrow 0$, 差分方程截断误并$R_j^n$对每一点$(x_j, t_n)$都趋近于0, 则该差分方程$(L\Delta u)_j^n=0$逼近微分方程$L\Delta u=0$, 差分方程与微分方程是相容的.
\item 收敛性: 节点$(x_p, t_p)$为偏微分方程求解区域$\Omega$内任意一点, 当$x\rightarrow x_p$, $t\rightarrow t_p$时, 差分方程数值解$u_j^n$逼近于微分方程精确解, 即$e_j^n=u-u_j^n=0$, 则差分方程收敛于该偏微分方程.
\item 稳定性: 在某一时刻$t_n$, 差分方程的计算误差为$\varepsilon_j^n$, 若在$t^{n+1}$时刻满足:
    \[
    ||\varepsilon_j^{n+1}|| \leq k||\varepsilon_j^{n}||
    \]
    则该差分方程是稳定的.
\end{itemize}
\vspace{1em}

\noindent 04. 写出Lax等价定理和Von Neumann准则基本内容.
\begin{itemize}
\item Lax等价定理: 对于适定和线性的偏微分方程的初值问题, 若逼近它的差分方程与它是相容的, 则差分方程的稳定性是保证差分方程收敛性的充分必要条件.
\item Von Neumann准则: 差分方程稳定性必要条件是当$\Delta t\leq \Delta t_0$时, 对于所有的波数$k$有:

    (1) $||G(\Delta t, \Delta x, k)||\leq 1 + K_1 \Delta t$\\
    (2) $\rho(G)=\max_j|\lambda_j(\Delta t, \Delta x, k)|\leq 1 + K_2\Delta t$

其中$\lambda_j(\Delta t, \Delta x, k)$为差分方程放大矩阵的特征值.
\end{itemize}

\noindent 05. 简要说明差分方程耗散性和色散性的主要特征和判别公式.
\begin{itemize}
\item 耗散效应: 差分方程计算激波时激波被拉宽, 幅度减小, 出现抹平和光滑现象.
\item 色散效应: 激波上下游出现高频振荡.
\item 修正方程截断误差的偶阶导数项为耗散项, 奇阶为色散项.
\item Founier分析法: 偏微分方程放大因子$G_e=|G_e|\mathrm{e}^{\varphi_e}$, 差分方程放大因子$G=|G|\mathrm{e}^{\varphi}$. 则: $\frac{|G|}{|G_e|}>1$负耗散, 不稳定; $\frac{|G|}{|G_e|}<1$正耗散, 稳定; $\frac{\varphi}{\varphi_e}>1$ 正色散, 相位是超前的; $\frac{\varphi}{\varphi_e}<1$ 负色散, 相位是滞后的.
\end{itemize}
\vspace{1em}

\noindent 06. 写出非线性发展方程间断形式和商条件, 并说明熵条件几何特性和物理特性.
\begin{itemize}
\item 间断形式: 在$\xi$处有间断$x_2<\xi(t)<x_1$, 则有$f^+-f^-=\frac{d\xi}{dt}(u^+-u^-)$.
\item 熵条件: $u(x,t)$是分片连续可微函数, 在连续区域内满足非线性演化方程$u_t+f_x=0(t>0, -\infty<x<+\infty)$和初始条件$u(x,0)=\phi(x)$, 间断线上满足
    \[
    \frac{f(u^{-})-f(u)}{u^{-}-u}
    \geq
    \frac{f(u^{+})-f(u^{-})}{u^{+}-u^{-}}
    \geq
    \frac{f(u^{+})-f(u)}{u^{+}-u}
    \]
\item 熵条件几何特性: 对于曲线$y=f(u)$, $u^+<u<u^-$. 由熵条件不等式左边得:
    在$(u^+,u^-)$内, $f(u)$位于($u^+,f^+$)和($u^-,f^-$)两点连线下方.
        由熵条件不等式左右得: 在$(u^-,u^+)$内, $f(u)$位于($u^+,f^+$)和($u^-,f^-$)两点连线上方.
\item 熵条件物理特性:
\end{itemize}
\vspace{1em}

\noindent 07. 什么是守恒方程和守恒差分格式是什么? 它们的相容条件是什么.
\begin{itemize}
\item 守恒方程: $u_t+f_x=0$, 初始条件$u(x,0)=\phi(x)$.
\item 守恒差分格式: $u_j^{n+1}=u_j^n-r(h_{j+1/2}^n-h_{j-1/2}^n)$, 其中$h_{j+1/2}^n=h^n(u_{j-l+1}^n, u_{j-l+2}^n, \cdots, u_{j+l}^n)$, $r=\Delta t/\Delta x$.
\item 相容条件: $h(w,w,\cdots,w)=f(w)$, 其中$w$是守恒型方程中的一个参数.
\end{itemize}
\vspace{1em}

\noindent 08. 什么是单调差分格式和保单调格式? 说明Godunov线性单调格式形式和性质.
\begin{itemize}
\item 单调差分格式: 用差分方程解$u_t+au_x=0$, $u(x,0)=\phi(x)$时, 若初始函数是单调非增(非减), 差分格式的解也是单调非增(非减).
\item 保单调格式:若非线性差分格式$u_j^{n+1}=G(u_{j-l}^n,\cdots, u_{j+l}^n)$在$n$时刻是单调非增(非减), 在$n+1$时刻$u_j^{n+1}$也是单调非增(非减).
\item Godunov线性单调格式形式及性质:
\[
u_j^{n+1} = \sum_k\alpha_ku_{j+k}^n
\]
定理: 对于初始条件为$u(x,0)=\phi(x)$的线性对流方程$u_t+au_x=0$:
\begin{enumerate}
\item 差分格式$u_j^{n+1} = \sum_k\alpha_ku_{j+k}^n$是单调格式的充要条件是所有的$\alpha_k\geq 0$.
\item 若$\sum_k\alpha_k=1$, 则单调差分格式是稳定的. 它的数值解收敛于物理解.
\item 单调差分格式是一阶精度, 并且解也保持单调性.
\end{enumerate}

\end{itemize}

\vspace{1em}
\noindent 09. 说明有限体积算法基本思想, 控制单元类型, 离散格式应遵循的四个原则.
\begin{itemize}
\item 基本思想: 首先把计算区域近似离散成有限个互不重叠的网格. 围绕每个网格点取一系列互不重叠的控制体单元, 在每个控制体单元中只包含一个节点. 并把待求流动量设置在网格节点上, 然后利用流动量守恒律对每个控制体单元进行积分, 导出一组离散格式. 对它进行求解, 得到流动的数值解.
\item 控制单元类型: 二维有三解形, 四边形或多边形; 三维有四面体, 锥形体, 楔体, 八面体等.
\item 四个原则:
       \begin{enumerate}
       \item 界面一致性. 相邻控制单元界面上流通量离散格式相同, 确保守恒律.
       \item 正系数原则. 离散格式中各项系数为正或同号, 确保稳定性和物理解.
       \item 负斜率原则. 源项线性化时, 系数小于零, 以确保误差不会越来越大.
       \item 系数和原则. 对无源流动, 中心节点系数为相邻节点系数和.$a_p=\sum a_{nb}$.
       \end{enumerate}
\end{itemize}
\vspace{1em}

\noindent 10. 说明有限体积算法和有限差分算法关系.

对于一维流动问题有限体积算法和有限差分算法完全等价; 对于多统流动问题, 只要控制体单元取矩形(2D), 长方体(3D), 它们基本上也等价. 但如果在多维问题中控制体单元不取矩形(2D), 长方体(3D), 它们并不完全等价.

%\newpage
%\lhead{2011年考试重点}                                                 %
\problemlist{\bf 高等计算流体力学2011年考试重点\footnote{{\bf 说明:} 这是2011年考试重点, 从一师兄那得来的, 仅供参考, 不代表本人观点.}}

\noindent \textbf{1. 等熵方程的数学性质. $\displaystyle \frac{dS}{dt}=0\:\Rightarrow\: \frac{\partial S}{\partial t}+u\frac{\partial S}{\partial x}=0$}
\vspace{1em}

\noindent \textbf{2. 对流方程和对流-扩散方程的数学物理性质.}\\
对流方程的数学物理性质:
\begin{itemize}
\item 它是双曲方程. 反映的流动特性可以用特征线来描述, 流场中扰动沿特征线传播, 在特征线上满足一定相容关系.
\item 它反映了流动中扰动传播的速度是有限的. 对于线性方程, 当$a>0$时, 扰动沿正方向; 当$a>0$时, 没负方向.
\item 扰动波幅不衰减, 波动不变化. 无耗散和色散.
\item 通解存在依赖区和影响区. 扰动传播范围有限, 不影响全流场.
\end{itemize}
扩散方程的数学物理性质:
\begin{itemize}
\item 它是抛物型方程, 有耗散无色散. 描述具有耗散机制的流动现象.
\item 任何局部扰动都会影响全流场.
\item 通解依赖于初始时刻粒子密度分布$\phi(\xi)$, 只要初值在$|x|<\infty$内连续有界, 则通解唯一, 连续的存在.
\item 它存在极值原理, 即如果初值是有界的, 且满足$m\leq \phi(x)\leq M$, 则通解也一定有界, 且满足$m\leq u(x,t)\leq M$
\end{itemize}
对流-扩散方程的数学物理性质:
\begin{itemize}
\item 它是双曲-抛物型方程. 解单值连续, 且永远存在.
\item 具有波动特性, 扰动特征线传播, 且传播速度是有限的.
\item 具有黏性流动特性, 有耗散无色散.
\end{itemize}
\vspace{1em}

\noindent \textbf{3. 相容性, 收敛性, 稳定性}
\begin{itemize}
\item 相容性: 对于足够光滑的函数, 若$\Delta x\rightarrow 0$, $\Delta t\rightarrow 0$, 差分方程截断误并$R_j^n$对每一点$(x_j, t_n)$都趋近于0, 则该差分方程$(L\Delta u)_j^n=0$逼近微分方程$L\Delta u=0$, 差分方程与微分方程是相容的. 
\item 收敛性: 节点$(x_p, t_p)$为偏微分方程求解区域$\Omega$内任意一点, 当$x\rightarrow x_p$, $t\rightarrow t_p$时, 差分方程数值解$u_j^n$逼近于微分方程精确解, 即$e_j^n=u-u_j^n=0$, 则差分方程收敛于该偏微分方程. 
\item 稳定性: 在某一时刻$t_n$, 差分方程的计算误差为$\varepsilon_j^n$, 若在$t^{n+1}$时刻满足:
    \[
    ||\varepsilon_j^{n+1}|| \leq k||\varepsilon_j^{n}||
    \]
    则该差分方程是稳定的.
\end{itemize}
\noindent \textbf{4. Lax等价定理和Von Neumann准则基本内容.}
\begin{itemize}
\item Lax等价定理: 对于适定和线性的偏微分方程的初值问题, 若逼近它的差分方程与它是相容的, 则差分方程的稳定性是保证差分方程收敛性的充分必要条件. 
\item Von Neumann准则: 差分方程稳定性必要条件是当$\Delta t\leq \Delta t_0$时, 对于所有的波数$k$有:
    
    (1) $||G(\Delta t, \Delta x, k)||\leq 1 + K_1 \Delta t$\\
    (2) $\rho(G)=\max_j|\lambda_j(\Delta t, \Delta x, k)|\leq 1 + K_2\Delta t$
    
其中$\lambda_j(\Delta t, \Delta x, k)$为差分方程放大矩阵的特征值.
\end{itemize}
\noindent \textbf{5. 差分方程耗散性和色散性的主要特征和判别公式.}

\noindent \textbf{6. 非线性发展方程间断形式, 和商条件表达式.}

\noindent \textbf{7. 守恒方程和守恒差分格式是什么? 二者相容条件.}

\noindent \textbf{8. 什么是单调差分格式? 保音调差分格式是什么?}

\noindent \textbf{9. 四项基本原则, 熵条件几何意义.}


\end{document} 
