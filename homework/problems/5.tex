\begin{problem}[问题5]
分析对流方程的紧致差分格式的精度和稳定性:
\[
u_j^{n+1} = u_j^n -rF_j^n
\]
\[
-\frac{1}{3}F_{j+1}^n + \frac{2}{3}F_j^n + \frac{2}{3}F_{j-1}^n
= -\frac{1}{2}\big(u_{j+1}^n-u_j^n\big) + \frac{3}{2}\big(u_j^n-u_{j-1}^n\big)
\]
\end{problem}

\begin{solution}
\textbf{精度分析:} 将$u_{j+1}^{n+1}$, $u_{j-1}^{n+1}$, $u_{j+1}^{n}$, $u_{j}^{n+1}$,
$u_{j-1}^{n}$分别在$u_{j}^{n}$处展开成泰勒级数
{\setlength\arraycolsep{2pt}
\begin{eqnarray*}
\textcolor{red}{u_{j+1}^{n+1}} & = & u_{j}^{n}+\Big(\Delta t\frac{\partial}{\partial t}+\Delta x\frac{\partial}{\partial x}\Big)u+\frac{1}{2}\Big(\Delta t\frac{\partial}{\partial t}+\Delta x\frac{\partial}{\partial x}\Big)^{2}u+\frac{1}{6}\Big(\Delta t\frac{\partial}{\partial t}+\Delta x\frac{\partial}{\partial x}\Big)^{3}u+\cdots\\
\textcolor{blue}{u_{j-1}^{n+1}} & = & u_{j}^{n}+\Big(\Delta t\frac{\partial}{\partial t}-\Delta x\frac{\partial}{\partial x}\Big)u+\frac{1}{2}\Big(\Delta t\frac{\partial}{\partial t}-\Delta x\frac{\partial}{\partial x}\Big)^{2}u+\frac{1}{6}\Big(\Delta t\frac{\partial}{\partial t}-\Delta x\frac{\partial}{\partial x}\Big)^{3}u+\cdots\\
\textcolor{red}{u_{j+1}^{n\hphantom{+1}}} & = & u_{j}^{n}+\frac{\partial u}{\partial x}\Delta x+\frac{1}{2}\frac{\partial^{2}u}{\partial x^{2}}\Delta x^{2}+\frac{1}{6}\frac{\partial^{3}u}{\partial x^{3}}\Delta x^{3}+O(\Delta x^{4})\\
\textcolor{blue}{ u_{j-1}^{n\hphantom{+1}}} & = & u_{j}^{n}-\frac{\partial u}{\partial x}\Delta x+\frac{1}{2}\frac{\partial^{2}u}{\partial x^{2}}\Delta x^{2}-\frac{1}{6}\frac{\partial^{3}u}{\partial x^{3}}\Delta x^{3}+O(\Delta x^{4})\\
\textcolor[rgb]{0.33,0.33,0.33}{u_{j}^{n+1}} & = & u_{j}^{n}+\frac{\partial u}{\partial t}\Delta t\:+\frac{1}{2}\frac{\partial^{2}u}{\partial t^{2}}\Delta t^{2}\:+\frac{1}{6}\frac{\partial^{3}u}{\partial t^{3}}\Delta t^{3}\:+O(\Delta t^{4}\:)
\end{eqnarray*}}
因此有
{\setlength\arraycolsep{2pt}
\begin{eqnarray*}
\textcolor{red}{\frac{u_{j+1}^{n+1}-u_{j+1}^{n}}{\Delta t}}
          & \textcolor{red}{=} & \textcolor{red}{\frac{\partial u}{\partial t}+\frac{1}{2}\frac{\partial^{2}u}{\partial t^{2}}\Delta t+\frac{\partial^{2}u}{\partial x\partial t}\Delta x
          +\frac{1}{6}\frac{\partial^{3}u}{\partial t^{3}}\Delta t^{2}+\frac{1}{2}\frac{\partial^{3}u}{\partial t\partial x^{2}}\Delta x^{2}+\frac{1}{2}\frac{\partial^{3}u}{\partial t^{2}\partial x}\Delta t\Delta x+\cdots}\\
\textcolor{blue}{\frac{u_{j-1}^{n+1}-u_{j-1}^{n}}{\Delta t}}
          & \textcolor{blue}{=} & \textcolor{blue}{\frac{\partial u}{\partial t}+\frac{1}{2}\frac{\partial^{2}u}{\partial t^{2}}\Delta t-\frac{\partial^{2}u}{\partial x\partial t}\Delta x+
          \frac{1}{6}\frac{\partial^{3}u}{\partial t^{3}}\Delta t^{2}+\frac{1}{2}\frac{\partial^{3}u}{\partial t\partial x^{2}}\Delta x^{2}-\frac{1}{2}\frac{\partial^{3}u}{\partial t^{2}\partial x}\Delta t\Delta x+\cdots}\\
\textcolor[rgb]{0.33,0.33,0.33}{\frac{u_{j}^{n+1}-u_{j\hphantom{-1}}^{n}}{\Delta t}}
          & \textcolor[rgb]{0.33,0.33,0.33}{=} & \textcolor[rgb]{0.33,0.33,0.33}{\frac{\partial u}{\partial t}+\frac{1}{2}\frac{\partial^{2}u}{\partial t^{2}}\Delta t\:+\frac{1}{6}\frac{\partial^{3}u}{\partial t^{3}}\Delta t^{2}\:+\cdots}\\
\textcolor{DarkGreen}{\frac{u_{j+1}^{n\hphantom{+1}}-u_{j\hphantom{-1}}^{n}}{\Delta x}}
          & \textcolor{DarkGreen}{=} & \textcolor{DarkGreen}{\frac{\partial u}{\partial x}+\frac{1}{2}\frac{\partial^{2}u}{\partial x^{2}}\Delta x+\frac{1}{6}\frac{\partial^{3}u}{\partial x^{3}}\Delta x^{2}+\cdots}\\
\textcolor[rgb]{0.6,0.3,0.1}{\frac{u_{j}^{n\hphantom{-1}}-u_{j-1}^{n}}{\Delta x}}
          & \textcolor[rgb]{0.6,0.3,0.1}{=} & \textcolor[rgb]{0.6,0.3,0.1}{\frac{\partial u}{\partial x}-\frac{1}{2}\frac{\partial^{2}u}{\partial x^{2}}\Delta x+\frac{1}{6}\frac{\partial^{3}u}{\partial x^{3}}\Delta x^{2}+\cdots}
\end{eqnarray*}}
将题中的差分格式合并成如下差分方程
\[
\frac{1}{3}\textcolor{red}{\frac{u_{j+1}^{n+1}-u_{j+1}^{n}}{\Delta t}}
-\frac{2}{3}\textcolor[rgb]{0.33,0.33,0.33}{\frac{u_{j}^{n+1}-u_{j}^{n}}{\Delta t}}
-\frac{2}{3}\textcolor{blue}{\frac{u_{j-1}^{n+1}-u_{j-1}^{n}}{\Delta t}}
=
-\frac{1}{2}a\textcolor{DarkGreen}{\frac{u_{j+1}^{n}-u_{j}^{n}}{\Delta x}}
+\frac{3}{2}a\textcolor[rgb]{0.6,0.3,0.1}{\frac{u_{j}^{n}-u_{j-1}^{n}}{\Delta x}}
\]
上式的左边和右边分别如下(仅保留了二阶项)
\begin{eqnarray*}
\text{左边} & = & +\frac{1}{3}\textcolor{red}{\Bigg[\hphantom{-}\frac{\partial u}{\partial t}+\frac{1}{2}\frac{\partial^{2}u}{\partial t^{2}}\Delta t+\frac{\partial^{2}u}{\partial x\partial t}\Delta x+\frac{1}{6}\frac{\partial^{3}u}{\partial t^{3}}\Delta t^{2}+\frac{1}{2}\frac{\partial^{3}u}{\partial t\partial x^{2}}\Delta x^{2}+\frac{1}{2}\frac{\partial^{3}u}{\partial t^{2}\partial x}\Delta x\Delta t\Bigg]}\\
 &  & -\frac{2}{3}\textcolor[rgb]{0.33,0.33,0.33}{\Bigg[\hphantom{-}\frac{\partial u}{\partial t}+\frac{1}{2}\frac{\partial^{2}u}{\partial t^{2}}\Delta t\hphantom{\,\,+\frac{\partial^{2}u}{\partial x\partial t}\Delta x\,}+\frac{1}{6}\frac{\partial^{3}u}{\partial t^{3}}\Delta t^{2}\hphantom{+\frac{1}{2}\frac{\partial^{3}u}{\partial t\partial x^{2}}\Delta x^{2}+\frac{1}{2}\frac{\partial^{3}u}{\partial t^{2}\partial x}\Delta x\Delta t\;}\Bigg]}\\
 &  & -\frac{2}{3}\textcolor{blue}{\Bigg[\hphantom{-}\frac{\partial u}{\partial t}+\frac{1}{2}\frac{\partial^{2}u}{\partial t^{2}}\Delta t-\frac{\partial^{2}u}{\partial x\partial t}\Delta x+\frac{1}{6}\frac{\partial^{3}u}{\partial t^{3}}\Delta t^{2}+\frac{1}{2}\frac{\partial^{3}u}{\partial t\partial x^{2}}\Delta x^{2}-\frac{1}{2}\frac{\partial^{3}u}{\partial t^{2}\partial x}\Delta x\Delta t\Bigg]}\\
 & = & \hphantom{-\frac{2}{3}\Bigg[}\!\!-\frac{\partial u}{\partial t}-\frac{1}{2}\frac{\partial^{2}u}{\partial t^{2}}\Delta t+\frac{\partial^{2}u}{\partial x\partial t}\Delta x-\frac{1}{6}\frac{\partial^{3}u}{\partial t^{3}}\Delta t^{2}-\frac{1}{6}\frac{\partial^{3}u}{\partial t\partial x^{2}}\Delta x^{2}+\frac{1}{2}\frac{\partial^{3}u}{\partial t^{2}\partial x}\Delta x\Delta t\hphantom{\Bigg]}\\
\\
\text{右边} & = & -\frac{1}{2}a\textcolor{DarkGreen}{\Bigg[\frac{\partial u}{\partial x}+\frac{1}{2}\frac{\partial^{2}u}{\partial x^{2}}\Delta x+\frac{1}{6}\frac{\partial^{3}u}{\partial x^{3}}\Delta x^{2}\Bigg]} +\frac{3}{2}a\textcolor[rgb]{0.6,0.3,0.1}{\Bigg[\frac{\partial u}{\partial x}-\frac{1}{2}\frac{\partial^{2}u}{\partial x^{2}}\Delta x+\frac{1}{6}\frac{\partial^{3}u}{\partial x^{3}}\Delta x^{2}\Bigg]}\\
 & = & a\frac{\partial u}{\partial x}-a\frac{\partial^{2}u}{\partial x^{2}}\Delta x+a\frac{1}{6}\frac{\partial^{3}u}{\partial x^{3}}\Delta x^{2}
\end{eqnarray*}
因此有
\begin{align*}
\text{右边-左边} & =a\frac{\partial u}{\partial x}-a\frac{\partial^{2}u}{\partial x^{2}}\Delta x+\frac{1}{6}a\frac{\partial^{3}u}{\partial x^{3}}\Delta x^{2}+\frac{\partial u}{\partial t}-\frac{\partial^{2}u}{\partial x\partial t}\Delta x+\frac{1}{6}\frac{\partial^{3}u}{\partial t\partial x^{2}}\Delta x^{2}+O(\Delta t,\Delta x\Delta t)\\
 & =\frac{\partial u}{\partial t}+a\frac{\partial u}{\partial x}-\Big(a\frac{\partial^{2}u}{\partial x^{2}}+\frac{\partial^{2}u}{\partial x\partial t}\Big)\Delta x+\frac{1}{6}\Big(a\frac{\partial^{3}u}{\partial x^{3}}+\frac{\partial^{3}u}{\partial t\partial x^{2}}\Big)\Delta x^{2}+O(\Delta t,\Delta x\Delta t)\\
 & =\frac{\partial u}{\partial t}+a\frac{\partial u}{\partial x}+O(\Delta t,\Delta x\Delta t)
\end{align*}
因此该差分格式在时间上是一阶精度, 空间上二阶精度.

\vspace{1em}
\noindent\textbf{稳定性分析:} 对差分方程中的各项作Fourier展开
$u_{j}^{n}=\sum_{k=1}^{\infty}\hat{u}_{j}^{n}\mathrm{e}^{\mathrm{i}jk\Delta x}$
并代入式差分方程得到它的每个分量的误差方程:
{\setlength\arraycolsep{2pt}
\begin{eqnarray*}
\frac{1}{3}(G-1)\mathrm{e}^{\mathrm{i}k\Delta x}-\frac{2}{3}(G-1)-\frac{2}{3}(G-1)\mathrm{e}^{-\mathrm{i}k\Delta x} & = & -\frac{1}{2}r(\mathrm{e}^{\mathrm{i}k\Delta x}-1)+\frac{3}{2}r(1-\mathrm{e}^{-\mathrm{i}k\Delta x})\\
 & \Downarrow\\
2(G-1)\Big(\mathrm{e}^{\mathrm{i}k\Delta x}-2-2\mathrm{e}^{-\mathrm{i}k\Delta x}\Big) & = & 3r(1-\mathrm{e}^{\mathrm{i}k\Delta x})+9r(1-\mathrm{e}^{-\mathrm{i}k\Delta x})\\
 & \Downarrow\\
2(G-1)\Big[-\Big(1-\mathrm{e}^{\mathrm{i}k\Delta x}\Big)+2\Big(1-\mathrm{e}^{-\mathrm{i}k\Delta x}\Big)-3\Big] & = & 3r(1-\mathrm{e}^{\mathrm{i}k\Delta x})+9r(1-\mathrm{e}^{-\mathrm{i}k\Delta x})
\end{eqnarray*}}
其中$G=\hat{u}_{j}^{n+1}/\hat{u}_{j}^{n}$为放大因子. 令$1-\mathrm{e}^{-\mathrm{i}k\Delta x}=z$, 则上式可化为
\[
G=\frac{(3r-2)z+(9r+4)\bar{z}-6}{2(2\bar{z}-z-3)}=\frac{az+b\bar{z}-6}{4\bar{z}-2z-6}
\]
其中$\bar{z}$为$z$的复共轭, $a=3r-2$, $b=9r+4$. 令$z\bar{z}=z+\bar{z}=2(1-\cos\theta)=s$($0\leq s\leq 4$)则有
{\setlength\arraycolsep{2pt}
\begin{eqnarray*}
|G|^2&=&G\bar{G} = \frac{(az+b\bar{z}-6)(a\bar{z}+bz-6)}{(4\bar{z}-2z-6)(4z-2\bar{z}-6)}\\
     &=&\frac{(a^{2}+b^{2})z\hat{z}+ab(\hat{z}+z)^{2}-2abz\hat{z}-6(a+b)(\hat{z}+z)+36}
     {36z\hat{z}-8(\hat{z}+z)^{2}-12(\hat{z}+z)+36}\\
     &=& \frac{(a-b)^2s+abs^{2}-6(a+b)s+36}{24s-8s^{2}+36}\\
\end{eqnarray*}}
显然上式分母大于0. 为比较分子与分母的大小, 将分子分母作差:
\begin{eqnarray*}
\text{分子-分母}&=&\Big((a-b)^2s+abs^{2}-6(a+b)s+36\Big) - (24s-8s^{2}+36)\\
                &=&(a-b)^2s+abs^{2}-6(a+b)s- 24s+8s^{2}\\
                &=&s\Big((a-b)^2+(ab+8)s-6(a+b)- 24\Big)\\
                &=&s\Big(36(r+1)^2+(27r^2-6r)s-12(6r+1)- 24\Big)\\
                &=&s\Big(36r^2+72r+36+27r^2s-6rs-72r-12- 24\Big)\\
                &=&3sr(12r+9rs-2s)
\end{eqnarray*}
因此$|G|\leq 1$成立的条件为
\[
12r+9rs-2s<0\:\Rightarrow\: r<\frac{2s}{12+9s} = \frac{2}{12/s+9}
\]
由于$0\leq s\leq 4$, 因此上述条件在$r>0$时不一定满足, 因此该差分格式是不稳定的.
%即$0<r=a\frac{\Delta t}{\Delta x}\leq 1/6$时, 该差分格式是稳定的.
\end{solution} 
