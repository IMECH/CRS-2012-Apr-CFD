\begin{problem}[问题2]
一维非定常粘性流动N-S方程组
\[
\mathbf{u_t} + \mathbf{f_x} = \mathbf{s_x}
\]
其中
\[
\mathbf{u} =
\left[
  \begin{array}{c}
    \rho \\
    \rho u\\
    E \\
  \end{array}
\right], ~~
\mathbf{f} =
\left[
  \begin{array}{c}
    \rho u \\
    \rho u^2 + p \\
    u(E+p) \\
  \end{array}
\right],~~
\mathbf{s} =
\left[
  \begin{array}{c}
    0 \\
    \frac{4}{3}\frac{\mu}{\textrm{Re}}\frac{\partial u}{\partial x} \\
    \frac{4}{3}\frac{\mu u}{\textrm{Re}}\frac{\partial u}{\partial x} + \frac{c_pK}{\textrm{Pr}~\textrm{Re}}\frac{\partial T}{\partial x} \\
  \end{array}
\right]
\]
这里的状态方程为:
\[
p = (\gamma -1)\rho e = (\gamma - 1)\Big(E-\frac{1}{2}\rho u^2\Big)
\]
求出该方程组的特征根, 并分析它的数学性质和类型.
\end{problem}

\begin{solution}
由状态方程可得
\[
E=c_{v}T+\frac{1}{2}\rho u^{2},\: u(E+p)=u\Big(c_{p} T+\frac{1}{2}\rho u^{2}\Big)
\]
因此上述守恒型方程组可改写成非守恒型方程组, 并消去能量和压强消项:
\[
\textcolor{red}{\frac{\partial\rho}{\partial t}}
+\textcolor{blue}{u\frac{\partial\rho}{\partial x}}
+\textcolor{DarkGreen}{\rho\frac{\partial u}{\partial x}} = 0
\]
\[
\textcolor{red}{\frac{\partial u}{\partial t}}
+\textcolor{DarkGreen}{u\frac{\partial u}{\partial x}}
+\textcolor{blue}{\frac{(c_p-c_v)}{\rho}\frac{\partial T}{\partial x}}
=\textcolor{blue}{\frac{4}{3}\frac{\mu}{\rho\textrm{Re}}\frac{\partial}{\partial x}\Big(\frac{\partial u}{\partial x}\Big)}
\]
\begin{multline}
\textcolor{red}{c_{v}\frac{\partial T}{\partial t}}
+\textcolor{DarkGreen}{\rho u\frac{\partial u}{\partial t}}
+\textcolor{DarkGreen}{\frac{1}{2}u^{2}\frac{\partial\rho}{\partial t}}
+\textcolor{DarkGreen}{c_{p} u\frac{\partial T}{\partial x}}
+\textcolor{DarkGreen}{c_{p} T\frac{\partial u}{\partial x}}
+\textcolor{DarkGreen}{\frac{1}{2}u^{3}\frac{\partial\rho}{\partial x}}
+\textcolor{DarkGreen}{\frac{3}{2}\rho u^{2}\frac{\partial u}{\partial x}}\\
\left. =
\textcolor{DarkGreen}{\frac{4}{3}\frac{\mu}{\textrm{Re}}\Big(\frac{\partial u}{\partial x}\Big)^{2}}
+\textcolor{DarkGreen}{\frac{4}{3}\frac{\mu}{\textrm{Re}}u\frac{\partial}{\partial x}\Big(\frac{\partial u}{\partial x}\Big)}+
\textcolor{blue}{\frac{c_{p}}{\textrm{Pr}}\frac{K}{\textrm{Re}}\frac{\partial}{\partial x}\Big(\frac{\partial T}{\partial x}\Big)}\right.\nonumber
\end{multline}
对上述非守恒型方程组进行适当变换, 令
\[
\frac{\partial u}{\partial x} = u_x, ~~ \frac{\partial T}{\partial x} = T_x
\]
可简化为下式
\[
\mathbf{A}\frac{\partial \mathbf{Z}}{\partial t} + \mathbf{B}\frac{\partial \mathbf{z}}{\partial x} = \mathbf{H}
\]
其中
\[
\mathbf{\mathbf{Z}}=[\rho\: u\: T\: u_{x}\: T_{x}]^{\textrm{T}}
\]
\[
A=\left[
\begin{array}{ccccc}
\textcolor{red}{1} & 0 & 0 & 0 & 0\\
0 & \textcolor{red}{1} & 0 & 0 & 0\\
0 & 0 & \textcolor{red}{c_{v}} & 0 & 0\\
0 & 0 & 0 & 0 & 0\\
0 & 0 & 0 & 0 & 0
\end{array}
\right]
\:\:\:\:\:\:\:\:\:\:\:\:\:\:\:\:\:\: \:\:\:\:\:\:\:\:\:\:\:\:\:\:\:\:\:\:\:\:\:\:\:
B=\left[
\begin{array}{ccccc}
\textcolor{blue}{u} & 0 & 0 & 0 & 0\\
0 & 0 & \scriptstyle \textcolor{blue}{\frac{(c_{p}-c_{v})}{\rho}} & \scriptstyle \textcolor{blue}{\frac{-4\mu}{3\rho\textrm{Re}}} & 0\\
0 & 0 & 0 & 0 & \scriptstyle\textcolor{blue}{\frac{-c_{p}K}{\textrm{Pr\textrm{Re}}}}\\
0 & 1 & 0 & 0 & 0\\
0 & 0 & 1 & 0 & 0
\end{array}
\right]
\]
\[
H=\left[\textcolor{DarkGreen}{
\begin{array}{c}
-\rho u_{x}\\
-uu_{x}\\
\frac{4}{3}\frac{\mu}{\textrm{Re}}(u_{x}^{2}+u_{xx})-\frac{3}{2}\rho u^{2}u_{x}-\frac{1}{2}u^{3}\rho_{x}-c_{p}(Tu_{x}+uT_{x})-\rho uu_{t}-\frac{1}{2}u^{2}\rho_{t}\\
u_{x}\\
T_{x}
\end{array}}
\right]
\]
系数矩阵的特征方程为
\[
\textrm{det}(\sigma_1 \mathbf{A} + \sigma_2 \mathbf{B}) = 0
\Rightarrow
(u\sigma_2+\sigma_1)\sigma_2^4=0
\]
它的特征根为:
\[
\lambda_1=u, ~~\lambda_{2,3,4,5} = 0
\]
除实根$\lambda_1$外, 其它均是重根. 由此可知, 该方程组是抛物型的, 但也具有双曲型性质. 从总体上说, 它是抛物型的.
\end{solution} 
